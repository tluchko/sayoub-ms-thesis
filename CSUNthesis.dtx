% \iffalse meta-comment
%%% The todo list:
%%% Tips and tricks
%%% spacing ties ~
%%
% \fi
% \CheckSum{0}
%
% \CharacterTable
%  {Upper-case    \A\B\C\D\E\F\G\H\I\J\K\L\M\N\O\P\Q\R\S\T\U\V\W\X\Y\Z
%   Lower-case    \a\b\c\d\e\f\g\h\i\j\k\l\m\n\o\p\q\r\s\t\u\v\w\x\y\z
%   Digits        \0\1\2\3\4\5\6\7\8\9
%   Exclamation   \!     Double quote  \"     Hash (number) \#
%   Dollar        \$     Percent       \%     Ampersand     \&
%   Acute accent  \'     Left paren    \(     Right paren   \)
%   Asterisk      \*     Plus          \+     Comma         \,
%   Minus         \-     Point         \.     Solidus       \/
%   Colon         \:     Semicolon     \;     Less than     \<
%   Equals        \=     Greater than  \>     Question mark \?
%   Commercial at \@     Left bracket  \[     Backslash     \\
%   Right bracket \]     Circumflex    \^     Underscore    \_
%   Grave accent  \`     Left brace    \{     Vertical bar  \|
%   Right brace   \}     Tilde         \~}
%
%^^A
%^^A List general changes here
%^^A
% \changes{2.3}{02 Nov 2020}
%   {Modified table of contents to temporarily have parskip=0.}
% \changes{2.2}{26 Apr 2006}
%   {Modified table of contents to temporarily have parskip=0.}
% \changes{1.4}{20 Apr 2005}
%   {Full conformance to 2004 guidelines achieved}
% \changes{1.4}{20 Apr 2005}
%   {Many typoos fixed}
% \changes{1.4}{20 Apr 2005}
%   {Documentation completed}
% \changes{1.3}{19 Apr 2005}
%   {Template examples completed}
% \changes{0.9}{04 Jan 2005}
%   {corrected use of wrong front page setup command in preface,
%   dedication and preface page commands}
% \changes{0.8}{20 Dec 2004}
%   {docstrip documentation added.}
% \changes{0.5}{03 Dec 2004}
%   {Brought into conformance with Graduate Evalutor's requirements.}
% \changes{0.1}{22 Jul 2004}
%   {Initial creation}
%
% \iffalse meta-comment
%
% Doc-Source file to use with LaTeX2e
%
% Copyright (C) 2020 by Jeff Wiegley, Ph.D.
% -----------------------------------------
%
% This file may be distributed and/or modified under the
% conditions of the LaTeX Project Public License, either
% version 1.2 or this license of (at your option) any
% later version. The latest version of this license is in:
%
%    http://www.latex-project.org/lppl.txt
%
% and version 1.2 or later is part of all distributions of
% LaTeX version 1999/12/01 or later.
%
% \fi
%
% \iffalse
%
%<*batchfile>
\begingroup

\input docstrip.tex
\keepsilent

\preamble

Copyright (C) 2020 by Jeff Wiegley, Ph.D.

This file may be distributed and/or modified under the
conditions of the LaTeX Project Public License, either
version 1.2 or this license of (at your option) any
later version. The latest version of this license is in:

   http://www.latex-project.org/lppl.txt

and version 1.2 or later is part of all distributions of
LaTeX version 1999/12/01 or later.

\endpreamble
\generateFile{CSUNthesis.cls}{f}{\from{CSUNthesis.dtx}{class}}

\endgroup
%</batchfile>
%
%<*dtx>
\ProvidesFile{CSUNthesis.dtx}
%</dtx>
%<*class>
%<class>\NeedsTeXFormat{LaTeX2e}
%<class>\ProvidesClass{CSUNthesis}
%<*class>
    [2020/11/01 2.3 CSU Northridge]
%</class>
%<*driver>
\documentclass{ltxdoc}
\EnableCrossrefs
\CodelineIndex
\RecordChanges
%\usepackage{times}%BLECH! cmr is a much easier to read typeface. Times is
%so heavy that it is almost bold. How did it ever get to be ubiquitous?
\usepackage{fancyvrb}
\usepackage{makeidx}
\usepackage{url}
\usepackage{lstdoc}
\usepackage{listings}
\usepackage{amssymb}
\usepackage{pifont}
\begin{document}
  \DocInput{CSUNthesis.dtx}
\end{document}
%</driver>
% \fi
%
% \GetFileInfo{CSUNthesis.dtx}
%^^A
%^^A I wanted to use the ``dbend'' symbol but it is defined in the
%^^A manfnt package which is not installed by default. It seems a
%^^A a lot of trouble to install an entire package for a single,
%^^A symbol.
%^^A
%\DeclareFontFamily{U}{manual}{}
%\DeclareFontShape{U}{manual}{m}{n}{ <->  manfnt }{}
%\newcommand{\manfntsymbol}[1]{%
%    {\fontencoding{U}\fontfamily{manual}\selectfont\symbol{#1}}}
% \newcommand{\dbend}{\manfntsymbol{127}}
% \newcommand{\danger}{\marginpar[\hfill\dbend]{\dbend\hfill}}
%
% \makeatletter
%^^A
%^^A Carsten Heinz produced truly excellent documentation for the
%^^A listings.sty package. His macros for syntax are helpful.
%^^A
%
%\def\@option{\begingroup
%   \catcode`\\12
%   \MakePrivateLetters \m@cro@ \iftrue {Macro}}
%\let\end@option\endmacro
%
%\def\@category{\begingroup
%   \catcode`\\12
%   \MakePrivateLetters \m@cro@ \iftrue {Category}}
%\let\end@category\endmacro
%
%\newenvironment{category}[1]{
%    \index{Code Categories\levelchar#1\actualchar\string\verb%
%     \quotechar*\verbatimchar#1\verbatimchar}%
%\begin{@category}{#1}}
%{\end{@category}}
%
%\newenvironment{option}[2]{
%    \index{#2\levelchar\string#1\actualchar\string\verb%
%     \quotechar*\verbatimchar\string#1\verbatimchar\encapchar usage}%
%    \index{\string#1\actualchar\string\verb%
%     \quotechar*\verbatimchar\string#1\verbatimchar\encapchar usage}%
%\begin{@option}{#1}}
%{\end{@option}}
%
%\newcommand{\indexbasic}[1]{
%  \texttt{#1}
%    \index{#1\actualchar\string\verb\quotechar*\verbatimchar#1\verbatimchar%
%           \encapchar usage}%
%}
%\newcommand{\indexslash}[1]{
%  \texttt{\string#1}
%    \index{\expandafter\@gobble\string#1\actualchar%
%           \string\verb\quotechar*\verbatimchar\string#1\verbatimchar%
%           \encapchar usage}%
%}
%\newcommand{\indexcommand}[2]{
%  \texttt{\string#2}
%    \index{Content commands\levelchar#1\levelchar%
%           \expandafter\@gobble\string#2\actualchar%
%           \string\verb\quotechar*\verbatimchar\string#2\verbatimchar%
%           \encapchar usage}%
%}
% \makeatother
%
%^^A italicindex
% \DoNotIndex{\#,\$,\%,\&,\@,\\,\{,\},\^,\_,\~,\ }
% \DoNotIndex{\@one}
% \DoNotIndex{\advance,\begingroup,\catcode,\closein}
% \DoNotIndex{\closeout,\day,\def,\edef,\else,\empty,\endgroup}
%
% \DoNotIndex{\@abstract,\@acknowledgement,\@author,\@chapapp}
% \DoNotIndex{\@collaborator,\@contact,\@coordinator,\@copyrightyear}
% \DoNotIndex{\@csunbibfile,\@csunbibstyle,\@dedication,\@defensedate}
% \DoNotIndex{\@defenseday,\@defenselocation,\@defensetime,\@degree}
% \DoNotIndex{\@department,\@highpenalty,\@makechapterhead}
% \DoNotIndex{\@makeschapterhead,\@memberA,\@memberB,\@memberChair}
% \DoNotIndex{\@plus,\@pnumwidth,\@preface,\@startsection,\@submitmonth}
% \DoNotIndex{\@submityear,\@tempdima,\@title,\AtBeginDocument}
% \DoNotIndex{\AtEndDocument,\ClassError,\ClassWarning,\ExecuteOptions}
% \DoNotIndex{\Large,\LoadClass,\MakeUppercase,\OptionNotUsed}
% \DoNotIndex{\PassOptionsToClass,\abstractfalse,\abstracttrue,\addpenalty}
% \DoNotIndex{\addtolength,\announcefalse,\announcetrue,\arabic,\baselineskip}
% \DoNotIndex{\bibliography,\bibliographystyle,\bibname,\c@secnumdepth}
% \DoNotIndex{\c@tocdepth,\centering,\draftfalse,\drafttrue,\evensidemargin}
% \DoNotIndex{\fontsize,\footskip,\frontmaatertopmargin,\frontpage,\hb@xt@}
% \DoNotIndex{\headheight,\headsep,\hfil,\hfill,\hskip,\ifabstract}
% \DoNotIndex{\ifannounce,\ifdraft,\iflof,\iflot,\ifproject,\ifproposal}
% \DoNotIndex{\ifthesis,\interlinepenalty,\item,\itemsep,\l@chapter}
% \DoNotIndex{\leavevmode,\leftmargin,\line,\listoffigures,\listoftables}
% \DoNotIndex{\loffalse,\loftrue,\lotfalse,\lottrue,\makebox,\maketitle}
% \DoNotIndex{\mpabstract,\mpacknowledgement,\mpannouncement,\mpcopyright}
% \DoNotIndex{\mpdedication,\mppreface,\mpproposal,\mpsignature}
% \DoNotIndex{\mptableofcontents,\mplistoffigures,\mplofillustrations}
% \DoNotIndex{\mplistoflistings,\mplistoftables,\mptitle,\newif,\newlength}
% \DoNotIndex{\newpage,\nobreak,\normalsize,\null,\oddsidemargin}
% \DoNotIndex{\oldbaselineskip,\oldheight,\oldoddside,\oldtop,\oldwidth}
% \DoNotIndex{\onehalfspacing,\p@,\pagenumbering,\pagestyle,\parindent}
% \DoNotIndex{\penalty,\projectfalse,\projecttrue,\proposalfalse}
% \DoNotIndex{\proposaltrue,\put,\section,\selectfont,\setcounter,\setlength}
% \DoNotIndex{\singlespacing,\subsection,\subsubsection}
% \DoNotIndex{\textbf,\textheight,\textwidth,\thesection,\thesisfalse}
% \DoNotIndex{\frontpagesetup,\thesistrue,\thispagestyle,\titlewidth,\today}
% \DoNotIndex{\topmargin,\topsep,\undefined,\vfill,\vskip,\vspace}
%
%^^A
%^^A Don't index TeX-primitives.
%^^A
% \DoNotIndex{\advance,\afterassignment,\aftergroup,\batchmode,\begingroup}
% \DoNotIndex{\box,\catcode,\char,\chardef,\closeout,\copy,\count,\csname,\def}
% \DoNotIndex{\dimen,\discretionary,\divide,\dp,\edef,\else,\end,\endcsname}
% \DoNotIndex{\endgroup,\endinput,\endlinechar,\escapechar,\everypar}
% \DoNotIndex{\expandafter,\fi,\gdef,\global,\globaldefs,\hbadness,\hbox}
% \DoNotIndex{\hrulefill,\hss,\ht}
% \DoNotIndex{\if,\ifdim,\iffalse,\ifhmode,\ifinner,\ifnum,\ifodd,\iftrue}
% \DoNotIndex{\ifvoid,\ifx,\ignorespaces,\immediate,\input,\jobname,\kern}
% \DoNotIndex{\lccode,\leftskip,\let,\long,\lower,\lowercase,\meaning,\message}
% \DoNotIndex{\multiply,\muskip,\noexpand,\noindent,\openout,\par,\parfillskip}
% \DoNotIndex{\parshape,\parskip,\raise,\read,\relax,\rightskip,\setbox,\skip}
% \DoNotIndex{\string,\the,\toks,\uppercase,\vbox,\vcenter,\vrule,\vtop,\wd}
% \DoNotIndex{\write,\xdef}
%
%^^A
%^^A Don't index LaTeX's private definitions.
%^^A
% \DoNotIndex{\@@end,\@@par,\@M,\@arabic,\@circlefnt,\@currentlabel}
% \DoNotIndex{\@currenvir,\@depth,\@doendpe,\@dottedtocline,\@eha,\@ehc}
% \DoNotIndex{\@empty,\@firstofone,\@firstoftwo,\@float,\@for,\@getcirc}
% \DoNotIndex{\@gobble,\@gobbletwo,\@halfwidth,\@height,\@ifnextchar}
% \DoNotIndex{\@ifundefined,\@ignoretrue,\@makecaption,\@makeother,\@namedef}
% \DoNotIndex{\@ne,\@noligs,\@notprerr,\@onlypreamble,\@secondoftwo,\@spaces}
% \DoNotIndex{\@starttoc,\@totalleftmargin,\@undefined,\@whilenum}
% \DoNotIndex{\@wholewidth,\@width}
% \DoNotIndex{\c@chapter,\c@figure,\c@page,\end@float,\f@family,\filename@area}
% \DoNotIndex{\filename@base,\filename@ext,\filename@parse,\if@twoside}
% \DoNotIndex{\l@ngrel@x,\m@ne,\new@command,\nfss@catcodes,\tw@,\thr@@}
% \DoNotIndex{\z@,\zap@space}
%
%^^A
%^^A Don't index LaTeX's package definitions.
%^^A
% \DoNotIndex{\AtEndOfPackage}
% \DoNotIndex{\CurrentOption,\DeclareOption,\IfFileExists,\InputIfFileExists}
% \DoNotIndex{\MessageBreak,\NeedsTeXFormat,\PackageError,\PackageWarning}
% \DoNotIndex{\ProcessOptions,\ProvidesFile,\ProvidesPackage,\RequirePackage}
%
%^^A
%^^A Don't index LaTeX's public definitions.
%^^A
% \DoNotIndex{\abovecaptionskip,\active,\addcontentsline,\addtocounter,\begin}
% \DoNotIndex{\belowcaptionskip,\bfseries,\bgroup,\bigbreak,\chapter}
% \DoNotIndex{\contentsname,\do,\egroup,\footnotesize,\index,\itshape}
% \DoNotIndex{\linewidth,\llap,\makeatletter,\newbox,\newcommand,\newcount}
% \DoNotIndex{\newcounter,\newdimen,\newtoks,\newwrite,\nointerlineskip}
% \DoNotIndex{\normalbaselines,\normalfont,\numberline,\pretolerance,\protect}
% \DoNotIndex{\qquad,\refstepcounter,\removelastskip,\renewcommand,\rlap}
% \DoNotIndex{\small,\smallbreak,\smallskipamount,\smash,\space,\strut}
% \DoNotIndex{\strutbox,\tableofcontents,\textasciicircum,\textasciitilde}
% \DoNotIndex{\textasteriskcentered,\textbackslash,\textbar,\textbraceleft}
% \DoNotIndex{\textbraceright,\textdollar,\textendash,\textgreater,\textless}
% \DoNotIndex{\textunderscore,\textvisiblespace,\thechapter,\ttdefault}
% \DoNotIndex{\ttfamily,\typeout,\value,\vphantom}
%
%^^A
%^^A Don't index definitions from other packages.
%^^A
% \DoNotIndex{\MakePercentComment,\MakePercentIgnore}
%
% \title{The CSU Northridge Masters Thesis \LaTeXe\ class\thanks{
%    This file has version \fileversion\ last revised \filedate}
% }
%
% \author{Jeff Wiegley, Ph.D. \\ \texttt{jeffw@csun.edu}\thanks{%
%    The class file author acknowledges the significant contributions
%    of Joohwan Lee, Amy Snetzler and Veerawan Sarbua for their
%    patience and invaluable assistance in pointing out and helping to
%    correct my initial blunders while creating the class file.}}
%
% \date{\filedate\ Version \fileversion}
%
% \maketitle
%
% \begin{abstract}
%    \LaTeXe\ is a professional document typesetting program. It is a
%    commonly accepted document preparation system used, or recognized,
%    by most of the mathematic and scientific disciplines. \LaTeX's
%    ability to handle and typeset mathematical equations properly and
%    with minimum effort is outstanding. Its knowledge and
%    implementation of professional
%    typesetting rules and behaviors is second to none. The biggest
%    advantage over alternative document applications is that \LaTeX\
%    allows an author to ignore document formatting, and concentrate
%    on the intellectual content instead.
%
%    The CSU Northridge Graduate Evaluator's office sets strict
%    guidelines regarding the visual appearance and format of graduate
%    theses. The \textsf{CSUNthesis}~\cite{url:csunthesis.cls} class file
%    was developed to enable
%    CSU Northridge graduate students to prepare their thesis using
%    \LaTeXe\ without having to learn the more advanced formatting
%    concepts used to control the details of \LaTeX. By
%    using this class file, graduate students need to learn only the
%    basics of \LaTeX. The use of this class file also allows graduate
%    students to concentrate on the content of their thesis topic and
%    almost completely ignore the format and layout of the
%    document.
%
%    The \textsf{CSUNthesis} class file correctly implements all the
%    format requirements of the Graduate Evaluator's office for the
%    preparation of graduate theses, projects and abstracts as
%    presented in the \emph{Guidelines for the preparation of
%    theses, graduate projects and artistic
%    abstracts}~\cite{url:2004csunthesesguide}
%
%    \textbf{\underbar{Please}}: Report any mistakes or problems with
%    the \textsf{CSUNthesis} class file to the class file author at
%    \texttt{jeffw@csun.edu} so that they may be permanently corrected.
% \end{abstract}
%
% \tableofcontents
%
% \parskip=10pt
% \parindent=0pt
%
% \lstset{basicstyle=\ttfamily\small,
%    numberstyle=\tiny,numbers=left,
%    xleftmargin=0.5in,
%    gobble=1,
%    language=Clean
%}
%
% \section{Introduction}
%
%    Impatient authors with prior knowledge of \LaTeX\ may want to
%    skip directly to Section~\ref{sec:reference} to begin using the
%    class file. Really impatient thesis authors who have procrastinated
%    until the night before their defense may want to skip to
%    Section~\ref{sec:templates}
%
%    \TeX\ is a markup language designed in 1978 by Donald
%    E. Knuth. \TeX\ is pronounced like ``tech'' as in \TeX-nician.
%    Donald Knuth is famous for writing \emph{The Art of Computer
%    Programming} books~\cite{knuth98:_art_progr}. While
%    writing the first volumes of his work he became frustrated by the
%    typesetting mistakes his publishers were making with his work and
%    with the limited abilities of the \emph{troff} program to produce
%    properly typeset results. This motivated him to write the Turing
%    complete language \TeX. The purpose of \TeX\ is to properly handle
%    the minute
%    details associated with properly formatting bits of text and
%    mathematical formulae.
%
%    \LaTeX\ was written by Leslie Lamport in 1985. \LaTeX\ is a
%    collection of macros, written in the language \TeX, that allows
%    an author to concentrate on the higher-level document features
%    rather than on the details of typesetting.
%
%    \TeX\ and \LaTeX\ are far older, more mature and more capable of
%    preparing professional technical documents than Microsoft Word,
%    WordPerfect or any other inferior graphical applications.
%
%    \subsection{Disadvantages of using \TeX/\LaTeX}
%    Using \TeX\ and \LaTeX\ is not without disadvantages:
%    \begin{itemize}
%      \item Documents are stored as plain 7-bit ASCII text and will
%      need to be compiled into a \texttt{.dvi}, \texttt{.ps} or
%      \texttt{.pdf} file in order to view the typeset results.
%
%      \emph{Instructions on how to do so are provided in this
%      documentation.}
%
%      \item The editing of documents is done through any ASCII
%      text editor and not through a ``What You See Is What You Get''
%      (WYSIWYG) graphical application.
%
%      \emph{This allows an author to choose what ever editor and
%      environment with which he is most productive.}
%
%      \item To include graphics into \LaTeX\ documents the
%      graphics need to be supplied as Encapsulated PostScript images.
%      \item The author needs to learn a new markup
%      language.
%
%      \emph{The core amount of information that is required though
%      fits in the first 40 pages of Lamport's informative book 
%      \emph{LaTeX: A Document Preparation
%      System}}~\cite{lamport94:_latex}.
%
%      \item Limited typeface capability. Switching typefaces is
%      not as easy. This is due to \TeX\ being developed at a time
%      when PostScript and TrueType typeface technologies were not
%      available. This limitation has been corrected and \TeX\ can
%      use TrueType typefaces, but it has to be setup properly and to
%      do so is not a trivial undertaking.
%   \end{itemize}
%
%   \subsection{Advantages of using \TeX/\LaTeX}
%   The presence of these disadvantages raises the question ``Why use
%   \LaTeXe?''
%   \begin{itemize}
%      \item It is free. You can obtain \TeX\ and \LaTeXe\ without cost
%      for just about any operating system available including Linux,
%      Windows, Mac and even DOS.
%
%      \item It is free. You can modify the source code to \LaTeXe\ and
%      even distribute the derived work under certain, reasonable
%      conditions.  If \LaTeXe\ doesn't do exactly what you want it to
%      then you can change it so that it does.
%
%      \emph{Though, doing so is not always trivial. The
%      \texttt{CSUNthesis.dtx} file is now $2,857$ lines and growing
%      just to provide thesis authors with the changes necessary to
%      conform to the thesis guidelines.}
%      
%      \item If you need to format mathematical formulae then \LaTeXe\
%      with the \texttt{amsmath} package is unbeatable.
%
%      Open up Microsoft Word's equation editor and try to insert the
%      simple formulae ``$B\nrightarrow A$''. (Hint: Word doesn't have
%      something as simple as negated arrows.) In \LaTeX\ that formula
%      is typeset by simply typing |$B\nrightarrow A$|.
%
%      A quick  survey of Word's equation editor reveals that Word
%      provides the author with a total of $286$ different symbols or
%      relationships; \LaTeXe\ provides an author with over $2,000$
%      symbols. The common math packages alone provide $576$
%      mathematical symbols (including the $\nrightarrow$ symbol
%      example). 
%
%      \item \LaTeX\ automatically organizes, numbers and builds your
%      \emph{Table of Contents}, \emph{List of Figures}, \emph{List of
%      Illustrations} and \emph{Bibliography} sections.
%      \item Figures, sections, chapters, subsections and equations
%      are all numbered automatically and updated whenever content is
%      reorganized. Cross-references to numbered material are also
%      automatically updated when numbering changes.
%      \item With a decent class file, such as \textsf{CSUNthesis},
%      all formatting is handled for you automatically.
%      \item \LaTeX\ is always consistent. Spacing between words, lines,
%      paragraphs, itemized or enumerated list items is always the
%      same and labeled consistently.
%      \item with a decent bibliography data file the content and
%      formatting of the bibliography section is always built
%      correctly. The bibliography is automatically sorted, formatted
%      and numbered whenever new citations are added to the document.
%      \item \LaTeX\ produces a document that is generally more
%      professional and ready for publishing than those produced by
%      weaker word processing systems.
%      \item \LaTeX\ is a professional document preparation system,
%      not a word processor.
%      \item \LaTeX\ supports \emph{literate programming}. This means
%      that the source file can be self documenting. Not only can
%      comments be inserted into the source file, but full
%      documentation can be supported. This documentation and the
%      \texttt{CSUNthesis.cls} file are generated from the same
%      original \LaTeXe\ source file.
%   \end{itemize}
%
%   The remainder of this document instructs an author on how to use
%   the \textsf{CSUNthesis} class class file to prepare a suitable
%   masters thesis using \LaTeX.
%
% \section{Academic (Dis)Honesty}
%
%   This section is present in order to inform students about the
%   standards to which they are required to adhere to when producing
%   an acceptable thesis. If you are reading this document because you
%   chose Prof. Wiegley or Prof. Noga and they are making you do you
%   thesis in \LaTeXe\ then you should at least be aware that both of
%   these professors, and almost all other professor's on campus, have
%   failed theses for reasons of academic dishonesty.
%
%   The first thing to remember is that, above all else, your thesis
%   stands as representing yourself to the world. Your thesis
%   represents the levels of excellence that you are capable of
%   acheiving and the depth to which your knowledge, research and
%   problem solving skills extend. What you produce will be archived
%   forever for the public to view.
%
%   Your thesis also reflects on the quality of education and the
%   reputation of CSU Northridge.
%
%   It is in nobody's interest to produce a thesis through any form of
%   academic dishonesty. It should consist of your ideas and your
%   analysis of the problem you tackled. The works of others should
%   only be included as evidence to support your conclusions.
%
%   The most common form of academic dishonesty is
%   \emph{plagiarism}. Plagiarism is is easy to commit and most
%   students dont' believe they have done anything wrong. Ignorance of
%   the law however is not an excuse. The end result of committing the
%   act will be that you are not allowed to defend.
%
%   So what is plagiarism and how do I avoid it?
%
%   Plagiarism: ``the act of plagiarizing; taking someone's words or
%   ideas as if they were your own''\cite{url:plagiarism}
%
%   Plagiarism occurs the moment you cut and paste the published
%   material produced by some other author without the following three
%   criteria:
%   \begin{enumerate}
%      \item you must have either the author's (or copyright holder's)
%      permission, or the amount and type of work being copied must
%      conform to the rules of fair use.\cite{url:fairuse}
%      \item You must cite the full bibliographic information for the
%      work in your references.
%      \item The item must be used as evidence to support a conclusion
%      stated by you. The item cannot be used as an alternative to
%      writing your ideas up, even you have the same concept to
%      state.
%   \end{enumerate}
%
%   For example, the definition of plagiarism included above has been
%   obtained via fair use rules (it's amount and scope is sufficiently
%   small.) I have also cited where the reference was found and who
%   published it. (It's a URL which are sometimes hard to find author
%   and date information for, but none the less it is cited.) I used
%   it as an example to demonstrate what it was I was talking about. I
%   did not put forth the notion that I was somehow defining
%   plagiarism myself.
%
%   \emph{``But somebody has a really great image I want to use and seems a
%   waste of time to build my own.''} Fine. Ask the author if you can
%   use his image. Most likely if all you want is an image they will
%   happily agree. Once permissions is obtained then you can cite the
%   origin of the image and provide your own analysis about the image
%   or draw conclusions from it.
%
%   \emph{``But they did some really great work that I want to talk about.''}
%   Lots of people do lots of great work. It is your responsibility to
%   be one of those people. To this end you should be generating your
%   own, unique, ideas. You ideas can be related to other author's
%   ideas or an outgrowth from them. In which case you can include
%   their work in order to compare, contrast and support their ideas
%   with what you have developed. For instance, take the following
%   excerpt:
%   \begin{quote}
%      After we came out of the church, we stood talking for some time
%      together of Bishop Berkeley's ingenious sophistry to prove the
%      nonexistence of matter, and that every thing in the universe is
%      merely ideal. I observed, that though we are satisfied his
%      doctrine is not true, it is impossible to refute it. I never
%      shall forget the alacrity with which Johnson answered, striking
%      his foot with mighty force against a large stone, till he
%      rebounded from it -- "I refute it thus."
%   \end{quote}\cite{oxford:boswelllife}
%   This is one of my favorite quotes and is illustrates an example of
%   an effective and easy method for proving that reality is real and
%   not some illusion present in some nebulous being. There exist
%   other effective arguments as well.
%
%   Notice that my point is that there exists argument to prove
%   reality and that there are certain types of quotes that I like
%   better than others. I have not used Mr. Johnson's quote, in place
%   of my own words, to state that reality exists. I used it to back
%   up my own concepts. You can use material in a similar fashion.
%
%   \emph{``But what I included is freely available on the web.''} That may
%   be true. The work you are borrowing could be ``freely'' available
%   for anybody to view but \emph{only} through the means and methods
%   that the publisher or author chose. You do not have the right to
%   expose their ideas through another means.
%
%   \emph{``But I only borrowed a little bit for the introduction; the rest
%   of the work is all mine.''} Too bad. If even just one paragraph is
%   plagiarized then the entire work unacceptable. It's sort of like
%   being pregnant... You can't be just a ``little'' pregnant.
%
%   If you have \textbf{any} doubts about whether or not an intended
%   course of action would constitute some form of academic dishonesty
%   then you should seek your advisor's advice. They will know what to
%   do and how to achieve the result you are looking for. It's one of
%   the reasons they exist for you.
%
% \section{Obtaining \LaTeXe}
%
%   \LaTeXe\ is available for almost every operating system. The method
%   for obtaining and installing \LaTeXe\ is a little different for each
%   operating system. Most authors will be working on a
%   \texttt{Linux} or \texttt{Microsoft Windows} operating system and
%   the general directions for these two systems are covered in this
%   document.
%
%   \subsection{Un*x}
%   Linux comes in several ``flavors'' called
%   distributions. \emph{RedHat}, \emph{Debian} and \emph{Slackware}
%   are some of the popular ones. Almost all Linux distributions are
%   packaged based these days, making installation of large
%   applications very easy.
%
%   under Debian the command:\begin{verbatim}
%       apt-get install tetex-base tetex-bin tetex-extras\end{verbatim}
%   will download and install all the necessary programs. Redhat users
%   can install similar RPM packages using tools provided by Redhat.
%
%   \subsection{Microsoft Windows}
%
%   Windows users will want to obtain and install the \LaTeXe\
%   distribution named \texttt{MikTeX}~\cite{url:miktex}.
%   
% \section{Document production}
%
%   Preparing a document using \LaTeX\ involves iteration of the
%   following steps.
%
%   \newlength{\oldleftmargini}
%   \setlength{\oldleftmargini}{\leftmargini}
%   \leftmargini=1in
%   \begin{enumerate}
%     \item[\textbf{Edit}:] Edit Source Code. Any \texttt{ASCII} text
%     editor is sufficient for this task. Any changes to the
%     document's content are made by editing a source file (or a collection
%     of source files).
%     \item[\textbf{Compile}:] Compile the source file using the
%     command \texttt{latex} (and possibly \texttt{bibtex}).
%     \item[\textbf{Preview}:] Preview the resulting document using a
%     preview application such as \texttt{xdvi} (Un$\star$x) or
%     \texttt{yap} (Windows).
%     \item[\textbf{Print}:] Convert the document to PostScript or PDF
%     for printing.
%
%     (The final review of a document should be performed on
%     the PostScript or PDF version using a viewer such as \texttt{xpdf}
%     or Acrobat Reader as \texttt{xdvi} and \texttt{yap} have some
%     problems rendering PostScript correctly.)
%   \end{enumerate}
%   \leftmargini=\oldleftmargini
%
% \subsection{Editing the document}
%
%    The source document is the file that the author works on and
%    changes. Traditionally, the extension of the file name is either
%    \texttt{.tex} or \texttt{.ltx} to reflect the file's purpose. The
%    source file is a simple 7-bit \texttt{ASCII} encoded file. Any
%    text editor can be used to create and edit the source
%    file. Common editors include \texttt{emacs}, \texttt{vi},
%    \texttt{pico}, \texttt{nano}, and \texttt{notepad}.
%
%    To illustrate the remaining procedures for creating the final
%    publishable document from the source file it is helpful to have a
%    sample document to reference. Listing~\ref{lst:example} on
%    Page~\pageref{lst:example} is such a document. Consider this to
%    be the source file named ``\texttt{example.tex}''.
%
%\begin{lstlisting}[
%    label=lst:example,
%    caption=Minimal CSUNthesis document example: \texttt{example.tex}
%]
%\documentclass[12pt]{CSUNthesis}
%
%\submitted{December}{2004}
%
%\author{John Doe}
%
%\title{CSUNthesis Example}
%
%\committee
%    {John Q. Public, Ph.D.}
%    {Jane Doe, Ph.D.}
%    {David Phantom, Ph.D.}
%
%\abstract{This document is a simple example of using the
%\textsf{CSUNthesis} class file.}
%
%\begin{document}
%
%\chapter{Introduction}
%This document was produced using \LaTeX\ code. The
%source for this example represents the minimum amount
%of markup commands needed in the preamble to produce
%a document using the \textsf{CSUNthesis} class
%file.
%
%\end{document}
%\end{lstlisting}
%
% \subsection{Compiling the document}
%
%    As the reader can see, the source file contains no binary
%    formatting. Instead formatting is specified by directives (called
%    \emph{macros}). The |\textsf{|$\ldots$|}| is a
%    directive that typesets its argument in \textsf{\underbar{s}ans-seri\underbar{f}}
%    typeface, a simplified stroke typeface similar to Helvatica.
%
%    To produce actual formatted output it is necessary to compile the
%    source file. The program \texttt{latex} is the compiler and
%    produces a formatted output file that ends in the filename suffix
%    of \texttt{.dvi}. DVI is an acronym for
%    \underbar{D}e\underbar{v}ice \underbar{I}ndependent file.
%
%    It is easy to compile simple files. The command\begin{verbatim}
%       latex example.tex\end{verbatim}
%
%    will compile the file and produce a DVI file as output. The
%    compilation also produces a variety of other auxiliary files which
%    will be discussed in a moment.
%
%    Authors should carefully read the debugging output that
%    \LaTeX\ issues while compiling a file. If it encounters any fatal
%    errors it will stop and fail to compile the file and tell the
%    author why. Non-fatal warnings will simply be logged and the
%    compilation will proceed.
%
%    The \danger worst warning that can occur that authors
%    \textbf{must} be aware of and correct are ``|Overfull \hbox|''
%    and ``|float too wide|'' warnings. \TeX\ uses the concept of
%    boxes to
%    construct the formatted output. Boxes
%    containing individual characters are glued together to make a box
%    containing a word, word boxes are glued together to make line
%    boxes and line boxes are glued together vertically to make
%    pages. An |Overfull \hbox| means that \TeX\ could not find a
%    decent place to break a line and has produced a box that is too
%    wide. This box extends out into the margin area and will result in
%    the result being rejected by the Graduate
%    Evaluator. \textbf{Always correct overfull hboxes!} This can
%    usually be done by simply altering the chosen wording for the
%    line. If the line is caused by a float figure being too wide then
%    the figure can be resized smaller to correct the problem.
%    \texttt{The~Graduate~Evaluator~will~not~accept~theses~that~violate~the~margins!}
%
% \subsubsection{Advanced compiling}
%
%    \LaTeX\ tries to perform its work in a single pass through the document. If
%    references, citations and section numbering is used then this
%    causes some problems. A citation or reference may be encountered
%    before the number is known. So \LaTeX\ is unable to determine the
%    proper value to produce in the output during the first pass.
%
%    The problem is solved through the generation of an auxiliary file
%    during the first pass. A second compilation pass is required and fixes up
%    the problem. The auxiliary file ends in a suffix of
%    \texttt{.aux}. Other, similar auxiliary files are created to
%    assist with the table of contents, list of figures and list of
%    tables if they are called for.
%
%    When \LaTeX\ computes a new number for a figure, section or
%    citation it records the label and number in the auxiliary
%    files. When \LaTeX\ encounters a reference to a figure, section
%    or citation it looks up the label in the auxiliary file and
%    obtains the value recorded earlier or in a previous pass.
%
%    The first compilation pass results in a complete and correct
%    auxiliary file but label references may not be correct since the
%    auxiliary file wasn't complete at the start of the pass. Rerunning
%    \texttt{latex} a second time will correct all the label
%    references.
%
%    The second pass is only necessary if label references have
%    changed. \LaTeX\ is pretty smart and will issue a warning at the
%    end of a compilation pass if it thinks it needs to be recompiled
%    to get the references corrected. In rare cases a third pass is
%    required.
%
%    In general a complete compilation run will consist of the
%    following commands:\begin{verbatim}
%       latex example.tex
%       bibtex example
%       latex example.tex
%       latex example.tex\end{verbatim}
%
%   \texttt{bibtex} only needs to be run when the bibliographic
%   database changes or when citations are changed in the source
%   document. The \BibTeX\ database and its usage is discussed in
%   Section~\ref{sec:bibtex}.
%
% \subsection{Previewing the document}
%
%   One drawback to the use of \LaTeX\ is its lack of ``What you see
%   is what you get'' interactive capabilities. After compiling the
%   file most authors will want to preview the formatted output to
%   make sure it is professional and ready for publication.
%
%   There are graphical previewers available that can handle DVI
%   files. On Un$\star$x like platforms with X-windows, the application
%   called \texttt{xdvi} can display the resulting DVI file. Under
%   windows the MikTeX distribution includes an application
%   named \texttt{yap}
%   (\underbar{y}et~\underbar{a}nother~\underbar{p}reviewer).
%
%   \texttt{xdvi} and \texttt{yap} both support recognition of a
%   change in the DVI file and redisplay the result. This means that
%   you only have to launch these programs once and then leave them
%   running. Every time the source file is recompiled the already opened
%   previewer will display the new result. (\texttt{xdvi} only checks
%   the file when the \texttt{xdvi} window is ``exposed''.)
%
%   \subsubsection{PostScript preview problems}
%
%   \texttt{xdvi} and \texttt{yap} are good, but both seem to have
%   problems previewing certain PostScript features. \texttt{xdvi},
%   for instance, is unable to display landscape rotated pages.
%
%   A fool-proof method for previewing documents is to convert the
%   DVI file to PostScript and then use a fully capable PostScript
%   previewing application. Conversion is done using the
%   \texttt{dvips} program as described in
%   Section~\ref{sub:convert}. \texttt{ghostscript} is a powerful
%   PostScript previewer that is free but it is not very user
%   friendly. \texttt{gv} (Ghost View) is a user friendly, paged
%   front-end to \texttt{ghostscript}. With the proper command line
%   argument \texttt{gv} also supports recognizing when the PostScript
%   file changes and updating the display.
%
% \subsection{Converting the document for printing}
% \label{sub:convert}
%   
%   Once the document is complete the author will need to print the
%   document. Modern printers typically communicate using proprietary
%   protocols and print languages and do not recognize DVI formatted
%   material natively. Many printers do understand the PostScript
%   printing language. Adobe produces \emph{Acrobat Reader}, a free
%   Portable Document Format (PDF) viewer that can print PDF files to
%   any printer. PostScript can easily be converted to PDF format.
%
%   PostScript makes a good common denominator. The \LaTeX\ packages
%   come with a DVI-to-PostScript converter named \texttt{dvips}
%
%   Running the command\begin{verbatim}
%       dvips -o example.ps example.dvi\end{verbatim} will produce
%   a PostScript formatted output
%   file named \texttt{example.ps} that is ready for printing or
%   conversion to PDF.
%
%   Conversion from DVI directly to PDF is possible using the
%   \texttt{dvipdf} tool. However, the author of the
%   \textsf{CSUNthesis} class has had problems with this tool where the
%   resulting format did not match that produced by \texttt{dvips}
%
%   A better way to produce to produce a PDF version is to use the
%   \texttt{ps2pdf}. The command\begin{verbatim}
%       ps2pdf example.ps example.pdf\end{verbatim} will produce a PDF
%    file named \texttt{example.ps} from the PostScript file
%    \texttt{example.ps}.
%    
%   So in conclusion the process is:
% $$ \textrm{.tex}\rightarrow \textrm{.dvi}\rightarrow \textrm{.ps}\rightarrow \textrm{.pdf}$$
%
% \subsection{Maintaining a bibliographic database}
% \label{sec:bibtex}
%
%   The bibliography is an essential component of any scholarly
%   work. It provides the reader with references to additional,
%   helpful information. It also provides previous authors with
%   recognition for the value of the work that they produced and that
%   the new document has relied on to achieve success. The correctness
%   and format of the bibliography is therefore very important and
%   great attention should be paid to its preparation.
%
%   The format of a bibliography is not standard. Some bibliographies
%   are sorted by author's last name, some are sorted by title. Some
%   italicize titles while others place the title in quotes. The
%   bibliographic style is usually dictated by the publisher. Famous
%   styles include the Chicago Manual of
%   Style~\cite{chicago03:_chicag_manual_style} and Kluwer Academic
%   publishers. With a variety of styles mandated by publishers,
%   authors do not want to have to re-format their bibliography
%   entries for each document. The problem is made worse by the fact
%   that authors frequently use the same references in different, but
%   related works and don't want to re-enter the information
%   repeatedly.
%
%   \LaTeX\ relies on a helper application named \BibTeX\ to automate
%   much of the tedious preparation required for professional
%   bibliography. \BibTeX\ solves these problems by having the author
%   maintain a single, format independent database file of
%   bibliographic information. The author maintains a single, separate
%   \texttt{ASCII} file that ends in a filename suffix of
%   \texttt{.bib}. This document assumes that the filename is
%   \texttt{authors.bib}.
%
%   Publications are listed in the \texttt{bib} file as specially
%   formatted entries (but a standard format that supports all
%   bibliographic styles in an independent
%   manner). Listing~\ref{lst:bibfile} illustrates the two
%   entries used to provide the Lamport~\cite{lamport94:_latex} and
%   Chicago Manual Of Style~\cite{chicago03:_chicag_manual_style}
%   references.
%
%\begin{lstlisting}[
%    label=lst:bibfile,
%    caption=\BibTeX\ database entry examples
%]
%@Book{lamport94:_latex,
%  author =       {Lamport, Leslie},
%  title =        {LaTeX: A Document Preparation System},
%  edition =      {third},
%  publisher =    {Addison-Wesley, Professional},
%  ISBN =         0201529831,
%  month =        jun,
%  year =         1994
%}
%
%@Misc{url:miktex,
%  title =        {{MikTeX}},
%  howpublished = {\url{http://www.miktex.org/}},
%}
%
%@Book{chicago03:_chicag_manual_style,
%  author =       {University of Chicago Press Staff},
%  title =        {The Chicago Manual of Style},
%  publisher =    {University of Chicago Press},
%  edition =      {$15^{th}$},
%  ISBN =         0226104036,
%  month =        aug,
%  year =         2003,
%  pages =        984
%}
%\end{lstlisting}
%
% Once the bibliographic database is maintained it needs to be
% incorporated into the compilation procedure. Whenever the
% \texttt{.bib} file is changed or if citations in the document are
% added or removed then \BibTeX\ needs to be rerun on the auxiliary
% file. If \LaTeX\ complains about ``undefined references'' then run
% the command\begin{verbatim}
%       bibtex example\end{verbatim} and then rerun the
% \texttt{latex} compiler until there are no cross-reference warnings.
%
% Various websites~\cite{url:bibtex1,url:bibtex2,url:bibtex3} can
% provide the author with more information and tutorials on using
% \BibTeX.
%
% \section{Tips and tricks}
%
% \subsection{Source code listings}
% In the discipline of computer science it is frequently necessary to
% illustrate source code. Carsten Heinz has written a very excellent
% package called \indexbasic{listings.sty}~\cite{url:listings} for
% typesetting source code into \LaTeX\ documents. Authors should
% refer to the PDF documentation available at the given URL for usage
% of this package.
%
% Authors should avoid cutting and pasting code into their
% thesis. Instead, use the \indexslash{\lstinputlisting} command from the
% |listings.sty| package to obtain the desired portion from an external
% file. This way if the source file changes, so does the thesis,
% thereby maintaining accuracy.
%
% \changes{1.1}{14 Apr 2005}
%   {Added section on Templates}
% \section{Templates}
% \label{sec:templates}
% The effort and experience required to modify \LaTeX\ to conform to
% the guidelines is rather extensive (nearly a thousand lines of
% \LaTeX\ code). This may make understanding the reference section
% rather overwhelming for the beginner. To make that learning curve
% easier the following subsections present basic templates for
% proposals and thesis creation.
%
% \subsection{The minimum requirements for a proposal}
% \label{sec:proposaltemplate}
% Graduate students begin with a proposal. The following shows example
% \LaTeX\ code used to create an acceptable proposal. Proposal
% format requirements are not as rigid as thesis requirements and are
% not subject to the approval of the Graduate Evaluator. The example
% sections in the proposal templates have been fleshed out with a
% description of the content graduate students should include in their
% proposal.
%
%\begin{lstlisting}[
%    label=lst:proposaltemplate,
%    caption=Proposal document example,
%    basicstyle=\small\ttfamily,
%]
%\documentclass[proposal]{CSUNthesis}
%
%\title{Minimum Proposal Example}
%
%\author{John Q. Public}
%\contact{jpublic@csun.edu}
%
%\committee{Jane Q. Vicktumb, Ph.D}
%          {Robert Victem, Ph.D}
%          {Ignacious Viktom, Ph.D}
%
%\coordinator{Richard O. Sight, Ph.D}
%
%\begin{document}
%
%  \section{OBJECTIVE}
%
%  A short statement describing the primary objective
%  of the proposed work. Shoot for no more than five
%  sentences.
%
%  \section{INTRODUCTION}
%
%  Introduce the reader to the area in which you will
%  be working. This includes basic definitions and
%  explanations geared toward an intelligent
%  undergraduate computer science student who has never
%  been exposed to your area of work. Proceed to
%  describe what sort of work has already been done and
%  exactly what you propose to do, going into as much
%  detail as possible. Indicate the technical importance
%  and/or interest of your proposed work. In discussing
%  previous work you should cite references gathered in
%  a literature search including critiques and summaries
%  of the references read and cited.
%
%  \section{TECHNICAL APPROACH}
%
%  Describe your approach to your work. For example,
%  describe any special machinery you will be using, the
%  design methods you plan to employ, insights you have
%  into potential new algorithms, etc. Produce a work
%  breakdown structure, listing the major tasks and
%  sub-tasks of the proposed project or thesis.
%
%  \section{SCHEDULE}
%
%  Provide a Gantt chart or similar device showing the
%  start and completion dates of the tasks and sub-tasks
%  described in the work breakdown structure.
%
%  \section{CRITERIA FOR SUCCESS}
%
%  Establish a set of criteria that can be used to
%  evaluate the degree of success of your proposed work.
%  For example, how much faster will your new algorithm
%  be than existing algorithms?  What level of skill
%  will your go playing program achieve?  How closely
%  will your simulator model the item being simulated
%  and how efficient will the simulation be?  It should
%  be possible to make an objective evaluation of your
%  completed work using the criteria you set down.
%
%  \references{plain}{references}
%
%\end{document}
%\end{lstlisting}
% \subsection{Minimum Thesis example}
% The same \textsf{CSUNthesis} class that was used to create the proposal
% is also used to create the full thesis. (Actually the class file's
% purpose is to create theses; the proposal format creation was added
% as a convenience. Proposals use only a small subset of the
% features provided by the class.)
%
% The biggest difference is the absence of \texttt{proposal} from the
% document class options.
%
% A template is presented here for illustrating the minimum macros
% requires for producing a thesis using the class file.
%\begin{lstlisting}[
%    label=lst:minimumtemplate,
%    caption=Proposal document example,
%    basicstyle=\small\ttfamily,
%]
%\documentclass{CSUNthesis}
%
%\title{A Minimum Thesis Example}
%
%\author{John Q. Public}
%
%\submitted{May}{2005}
%
%\committee{Jane Q. Vicktumb, Ph.D}
%          {Robert Victem, Ph.D}
%          {Ignacious Viktom, Ph.D}
%
%\abstract{This document in the result of a minimal
%          \textsf{CSUNthesis} thesis document.}
%
%\begin{document}
%
%  \chapter{Introduction}
%
%  Authors can use the minimum thesis template as a
%  starting point for creating their CSU Northridge,
%  masters thesis using \LaTeX\ and the
%  \textsf{CSUNthesis} class. The class file
%  automatically satisfies almost all of the
%  requirements as set forth in the ``Guidelines for
%  the preparation of theses, graduate projects and
%  artistic abstracts.''
%
%  Authors who wish to use the more advanced features
%  of the \textsf{CSUNthesis} class should refer to
%  the advanced template instead.
%
%\references{plain}{references}
%
%\end{document}
%\end{lstlisting}
%
%\subsection{A Maximum Thesis Example}
%
% Most authors will want to produce a thesis that consists of more
% than the minimum features. This example illustrates almost all of
% the features made possible by the \textsf{CSUNthesis} class
% (including the inclusion of source documentation).
%
%\begin{lstlisting}[
%    label=lst:maximumtemplate,
%    caption=Proposal document example,
%    basicstyle=\small\ttfamily,
%]
%\documentclass[10pt,lof,lot,lol]{CSUNthesis}
%
%%most CS students will want to present code listings.
%\usepackage{listings}
%
%%For easy line spacing
%\usepackage{setspace}
%
%%most authors will want to include graphics
%\usepackage{graphicx}
%
%\title{A Maximum Thesis Example}
%
%\author{John Q. Public}
%
%\submitted{May}{2005}
%
%\committee{Jane Q. Vicktumb, Ph.D}
%          {Robert Victem, Ph.D}
%          {Ignacious Viktom, Ph.D}
%
%\abstract{This document in the result of a maximal
%          \textsf{CSUNthesis} thesis document.}
%
%% There is more to life than computers
%\degree{Master of Science}{Psychology}
%
%%various front matter pages, the absence of any of these
%%macros just causes an absence of those pages.
%
%%information for producing a defense announcement page
%%comment out to supress announcement page.
%\defense{Monday}{May $14^\textrm{th}$}{3:30PM}{EA1440}
%
%%Cause a copyright page to appear after the title page.
%\copyrightyear{2005}
%
%\dedication{This template is dedicated to the brave,
%            pioneering students who choose to produce
%            their thesis using \LaTeX.}
%
%\acknowledgement{Special thanks to Joohwan Lee, Amy
%                 Snetzler and Joel Iniguez for pointing
%                 out all the mistakes present in the
%                 \textsf{CSUNthesis} class file and for
%                 their continued pressure to make the
%                 class file perfekt and well documented.}
%
%\preface{Authors may want to describe, or inform the
%         reader of something special prior to
%         presenting the thesis material.}
%
%\collaboration{Jimmy Anonymous}
%
%\begin{document}
%
%\chapter{Introduction}
%  This is an example including most of the features
%  students may want to use in their thesis.
%
%\section{First section}
%  Authors will want to research the following \LaTeX\
%  environments for producing certain formats:
%  \begin{itemize}
%  \item \verb|\begin{itemize}| and
%    \verb|\begin{enumerate}| for producing bulleted and
%    enumerated lists (like this one).
%  \item \verb|\begin{figure}| and
%    \verb|\begin{tabular}| for making figure and tabular
%     environments.
%  \item \verb|\chapter|, \verb|\section|,
%    \verb|\subsection| and \verb|\subsubsection| for
%    sectioning their work.
%  \item \verb|\begin{center}| for centering their
%     figures and tables.
%  \end{itemize}
%
%\section{Some other goodies}
%  Authors may want to change typefaces every now and
%  then to provide contextual information through the
%  use of typographical conventions.
%  \begin{itemize}
%  \item \verb|\texttt{desired text}| yields a
%    \texttt{fixed-width}, Courier-like typeface. Good
%    for use in writing out \texttt{method()} names,
%    code examples and \texttt{filenames}.
%  \item \verb|\emph{desired text}| produces
%    \emph{italics}. Good for introducing new
%    terminology. The vocabulary chosen to represent
%    something is called \emph{nomenclature}.
%  \item \verb|\textbf{desired text}| produces
%    \textbf{bold face}. Good for adding stength to a
%    portion of a statement. It \textbf{should} be used
%    carefully.
%  \item \verb|\textsf{desired text}| produces a
%    \textsf{sans-serif} typeface which is possibly good
%    for other typographical conventions. This document
%    uses sans-serif when referring to \LaTeX\ packages
%    such as \textsf{CSUNthesis}; though it
%    should be \texttt{CSUNthesis.cls} when referring to
%    the actual filename, and \textsf{CSUNthesis} when
%    referring to the class in general.
%  \end{itemize}
%
%%references come before appendicies.
%\references{plain}{references}
%
%\appendix % switch chapters to be appendicies
%
%\chapter{First appendix}
%
%\chapter{Last appendix}
%
%\end{document}
%\end{lstlisting}
%
% Since the example maximum template has a lot of hints
% provided in the example sections it might be beneficial
% to also provide the reader with the resulting format
% that would be produced by those sections to make it easier to
% read and digest and to see the product of the markup macros.
%
%\newsavebox{\myboxa}
%\begin{lrbox}{\myboxa}\verb|\section{First section}|\end{lrbox}
%\newsavebox{\myboxb}
%\begin{lrbox}{\myboxb}\verb|\section{Some other goodies}|\end{lrbox}
%\newsavebox{\myboxc}
%\begin{lrbox}{\myboxc}\verb|\begin{itemize}|\end{lrbox}
%\newsavebox{\myboxd}
%\begin{lrbox}{\myboxd}\verb|\begin{enumerate}|\end{lrbox}
%\newsavebox{\myboxe}
%\begin{lrbox}{\myboxe}\verb|\begin{figure}|\end{lrbox}
%\newsavebox{\myboxf}
%\begin{lrbox}{\myboxf}\verb|\begin{tabular}|\end{lrbox}
%\newsavebox{\myboxg}
%\begin{lrbox}{\myboxg}\verb|\chapter|\end{lrbox}
%\newsavebox{\myboxh}
%\begin{lrbox}{\myboxh}\verb|\section|\end{lrbox}
%\newsavebox{\myboxi}
%\begin{lrbox}{\myboxi}\verb|\subsection|\end{lrbox}
%\newsavebox{\myboxj}
%\begin{lrbox}{\myboxj}\verb|\subsubsection|\end{lrbox}
%\newsavebox{\myboxk}
%\begin{lrbox}{\myboxk}\verb|\texttt{desired text}|\end{lrbox}
%\newsavebox{\myboxl}
%\begin{lrbox}{\myboxl}\verb|\emph{desired text}|\end{lrbox}
%\newsavebox{\myboxm}
%\begin{lrbox}{\myboxm}\verb|\textsf{desired text}|\end{lrbox}
%\newsavebox{\myboxn}
%\begin{lrbox}{\myboxn}\verb|\begin{center}|\end{lrbox}
%\newsavebox{\myboxo}
%\begin{lrbox}{\myboxo}\verb|\textbf{desired text}|\end{lrbox}
%\hspace{0.25in}\fbox{\begin{minipage}{4.5in}
%\usebox{\myboxa}
%
%  Authors will want to research the following \LaTeX\
%  environments for producing certain formats:
%  \begin{itemize}
%  \item \usebox{\myboxc} and \usebox{\myboxd}
%    for producing bulleted and enumerated lists (like this one).
%  \item \usebox{\myboxe} and \usebox{\myboxf} for
%    making figure and tabular environments.
%  \item \usebox{\myboxg}, \usebox{\myboxh}, \usebox{\myboxi}
%    and \usebox{\myboxj} for sectioning their work.
%  \item \usebox{\myboxn} for centering their figures and
%     tables.
%  \end{itemize}
%
%\usebox{\myboxb}
%
%  Authors may want to change typefaces every now and then
%  to provide contextual information through the use of
%  typographical conventions.
%  \begin{itemize}
%  \item \usebox{\myboxk} yields a
%    \texttt{fixed-width}, Courier-like typeface. Good for
%    use in writing out \texttt{method()} names, code examples
%    and \texttt{filenames}.
%  \item \usebox{\myboxl} produces \emph{italics}.
%    Good for introducing new terminology. The vocabulary chosen
%    to represent something is called \emph{nomenclature}.
%  \item \usebox{\myboxo} produces \textbf{bold face}.
%    Good for adding stength to a portion of a statement. It
%   \textbf{should} be used carefully.
%  \item \usebox{\myboxm} produces a
%    \textsf{sans-serif} typeface which is possibly good for
%    other typographical conventions. This document uses
%    sans-serif when referring to \LaTeX\ packages such as
%    \textsf{CSUNthesis}; though it should be
%    \texttt{CSUNthesis.cls} when referring to the actual
%    filename, and \textsf{CSUNthesis} when referring to the
%    class in general.
%  \end{itemize}
%\end{minipage}}
%
% \section{Writing guidelines/FAQ}
%
% \subsection{The written vs spoken language}
%
% We learn to speak before we learn to read or write. While almost
% adults become fluent in a language there are great many that remain
% illiterate. Once we learn how to speak and understand what is spoken
% why can't we automatically read and write? There are least two reasons:
% \begin{enumerate}
% \item Reading and writing require the additional ability of
% recognizing and producing visual symbols.
% \item Speech is produced in real time and is
% continuous.\label{item:real-time}
% \end{enumerate}
% The more important point is item~\ref{item:real-time}. Speech is
% produced in real time and can be continuous. Any misunderstandings or omissions can
% be corrected and clarified by augmenting and continuing the
% spoken dialog. Written documents, on the other hand, are produced once and
% become a permanent, static transfer of knowledge. It is not possible
% to interact with the reader and make clarifications or corrections
% once the document has been published.
% It is therefore much harder to write than it is to speak because
% much more effort must be spent planning, designing and organizing
% the information that the author wishes to publishes so that
% corrections and clarifications are not required for readers to
% understand.
%
% Must students, and many authors, ignore how important this is. When
% writing you must be very specific, very consistent and very
% accurate. How long does it take to write a thesis? Well, for the
% first draft ten pages per day is rather easy. So why doesn't it take
% one week to write a thesis? Because the editing and organization to
% take a draft copy to final publication is $90\%$ of the effort. You can
% expect to spend nine times as much time editing your document as it
% took to include the majority of the content.
%
% \subsection{Providing examples}
%
% One failing that many authors make is the overuse of the
% abbreviation \emph{etc.} to indicate to the reader that a list of
% examples is not exhaustive. When speaking a list of examples to an
% audience you may find that in retrospective you need to indicate
% that the list is non-exhaustive. One can either add a phrase such as
% ``or other things such as those I have already mentioned'', or to
% make the converstation shorter simple say ``etcetera.''
%
% In writing however it is a rather lazy method of indicating a
% non-exhaustive list and indicates that the author has not taken the
% time necessary to properly organize his work. There is almost always
% a more professional alternative, in terms of grammar, that
% eliminates the need for this latin abbreviation and makes the
% reading pleasant.
%
% \emph{``There are many types of types of atoms. Cesium, Iron,
% Mercury, etc. to name a few''} should be replaced by \emph{``There
% are many types of atoms such as Cesium, Iron or Mercury.''} Thereby
% avoiding the use of etcetera. An equally suitable alternative is
% \emph{``There are many types of atoms including Cesium, Iron and
% Mercury.''} (Note the switch to \emph{and} instead of \emph{or} which
% is logically important.)
%
% Avoid the use of the word ``like'' when providing examples.  The
% word like either indicates a preference, as used in ``I like ice
% cream'', or indicates direct comparison, as used in ``The cat rose
% up on its hind legs like an angry bear.'' The cat is acting like a
% bear but the bear is not an example of a cat.
%
% \subsection{Don't dictate commands to the reader}
%
% Many thesis writers are inclined to describe the implementation of
% their work as a tutorial which leads the reader through the creation
% or use of their project. A thesis is a publication of knowledge and
% concepts learned; it is not a usage tutorial.
%
% Avoid, at all costs, commanding wording such as: ``Create a new
% APS.NET web 
% application and rename the default file WebForm1.aspx to
% ItemInfo.aspx''~cite{thesis:nmathews}. In doing so, you have commanded the reader to
% perform an action that they cannot possibly complete. The thesis
% presently
% in their hand is made of paper and is
% not a computer. It does not have the capability of allowing the
% reader to create a new web application nor does it contain a
% filesystem. Maybe, someday in the future, this will change but for
% now stick to providing the reader with digestible information and
% avoid telling them to do something.
%
% In general, writing a sequence of actions to perform to complete a task
% makes for some of the worst, most boring, reading when formatted as
% a narrative paragraph. If you must provide such information then
% provide it as an enumerated list where each item represents a single
% step. For instance, The steps a user would need to recreate the web
% appliction presented in this section consists of:
% \begin{enumerate}
% \item Using the IDE to create a new ASP.NET web application.
% \item renaming the default file, ``WebForm1.aspx'', to
% ``ItemInfo.aspx''.
% \item $\ldots$
% \end{enumerate}
% If you need to do this frequently in your thesis then you
% should first pick up an automotive factory service manual and see
% how professional procedural tutorials are written. First, you should
% probably consider that you are actually not writing a thesis but are
% instead writing an automotive service manual.
%
% \subsection{Consistency}
%
% Writing is an art. Like fine art, any brush stroke that
% inconsistent, or contrary, to the other strokes diminishes the
% quality of the work. Even a single wrong stroke can utterly destroy
% the beauty of the work.
%
% In writing one must be very careful to be consistent. There are many
% styles and formats to choose from.
% \subsubsection{Paragraph indentation and spacing}
% The indentation of
% paragraphs and the spacing between paragraphs is a choice made by
% the author. Here are some possibilities:
% \begin{itemize}
% \item Indent the first line of every paragraph, No spacing between
% paragraphs. (The beginning of a paragraph is easy to locate by it's
% indented line.)
% \item Paragraphs are not indented but spacing is added between
% paragraphs. (The beginning of a paragraph is easy to locate by the
% separation between paragraphs.)
% \item The first line of every paragraph (except the first paragraph
% of any chapter, section or subsection) is indented and spacing is added between
% paragraphs. (The beginning of a paragraph is easy to locate by the
% separation between paragraphs. Indentation is not necessary for the
% first paragraph because it can quickly be recognized because it
% follows a sectioning break.)
% \end{itemize}
% So which do you choose? It doesn't matter as long as you are
% perfectly consistent with its usage throughout the entire
% document.
%
% \subsubsection{Typographical conventions}
%
% Words, syntax and grammar are only some of the tools available to a
% writer to convey content. Sometimes these alone are not enough to
% convey the proper content to a reader or may become confusing
% without additional verbage or tools.
%
% Take, for example, the following discussion of source code:
% \begin{quote}
%    We did this to increment the variable classification by four.
% \end{quote}
% Does the author mean that there is a classification of variables and
% that they are promoting the topic to a higher classification? No. In
% this case the author is talking about a variable named
% ``classification.'' How can we clarify this for the reader? The most
% obvious method is to add wording or syntax to the phrase:
% \begin{quote}
%    We did this to increment the variable named ``classification'' by four.
% \end{quote}
% However, the use of double quotes is well established in writing to
% indicate either a direct quote of somebody else, to indicate that
% dialog between characters is occuring, or to indicate that the
% reader should consider a non-obvious connotation for a word. They
% should not be used to indicate a different context for word.
%
% See, you actually already knew about some well established and
% ubiquitous typographical conventions. You probably just didn't know
% that they were just part of a larger family or that you get to
% choose what the conventions are.
% 
% There are much better ways to provide contextual information for a
% word or phrase and these methods are termed \emph{typographical
% conventions}. (Notice that quotes were not used to provide the
% context that typographical conventions is a new phrase used to
% describe the class of concepts being introduced in this section.)
%
% Typographical conventions consist of assigning specific typefaces to
% represent additional context.
%
% \emph{Emphasis} is frequently used to indicate that a term is being
% introduced or defined.
%
% \texttt{Monospacing} is frequently used to indicate that source code
% or computerized information is being presented. So the best way to
% rework the above example is:
 % \begin{quote}
%    We did this to increment the variable \texttt{classification} by
%    four.
% \end{quote}
% This makes it shorter for the reader who is, presumably, use to
% encountering this convention while reading your document.
%
% \textsf{Sans-serif} is another style that can be used to indicate
% context. For computer science it can be used to indicate class names
% or applications.
%
% \textsc{Small caps} can also be used for another context. It is
% rather like \textsc{shouting} however, so it should be used for a context
% that is only needed infrequently.
%
% \textbf{Bold face} can be used to distinguish user input from output
% suppied by the system in response to the user's interaction.
%
% \underbar{underline} is another example. Though it, like bold face,
% is frequently used to indicate specific importance of a word or
% concept.
%
% When choosing your typographic conventions it is important to:
% \begin{enumerate}
% \item Choose carefully. Avoid choosing a convention that could lead to
% the reader having to guess between contexts. (Bold face for code for
% example could make it difficult for the reader because they might
% not be able to tell if you are illustrating a variable name or an
% important word.)
% \item Choose wisely. Try to choose conventions to match that which
% the reader might already be familiar with. For example, almost all
% program code editors and terminals display source code using a
% monospaced font such as \texttt{Courier}. When presenting code the
% reader you should also pick \texttt{Courier} to present your code
% samples so that reader encounters a familiar and friendly
% environment in which to read. To professional programmers source
% code looks absolutely hideous when presented in a proportional font
% where columns do not necessarily line up.
% \item Be thoroughly consistent. Never fail to use a convention that
% you have decided on and never use the convention for another purpose
% where that purpose could be confusing or where the convention's
% meaning could not be easily deduced from the surrounding context.
% \item Define your choice. Unless your selection is trivial or
% common, you should provide the reader with a
% preface section to your work that defines and illustrates what your
% conventions are. This is especially important if your conventions
% are non-traditional. Nearly every O'Reilly publication from their
% famous set of technical reference books includes such a preface
% section titled ``Conventions used in this book.''
% \end{enumerate}
%
% Some other typographical conventions include:
% \begin{itemize}
% \item Special formatting. Large quotes are indicated by increased
% left and right margins to produce a block quote.
% \item Increased indenting can be used to offset code or printouts
% from the main body. (Though it should still be in a monospaced
% typeface.)
% \item \danger Marginal icons can be used! (Though its use here is a bit
% erroneous since there is nothing dangerous about this item unless
% you are reading this while operating heavy machinery.)
% \end{itemize}
%
% \subsection{Quick tips}
% \begin{itemize}
%   \item \textbf{etc.} ends a list as though it were another item. It
%   is always preceded by a comma. (Though maybe you should avoid is
%   entirely.)
%   \begin{itemize}
%     \item[\ding{52}] He could dance, sing, fly, etc.
%     \item[\ding{56}] He could dance, sing, fly etc.
%   \end{itemize}
%   (Notice that the complete absence of \emph{etc.} in this case is
%   perfectly acceptable as well and would simplify the wording
%   without loss of content or meaning.)
%   \item \textbf{etc.} is an abbreviation of the word
%   ``etcetera''. Abbreviations are terminated with a period to
%   indicate the absent letters.
%   \begin{itemize}
%     \item[\ding{52}] He could dance, sing, fly, etc.
%     \item[\ding{56}] He could dance, sing, fly, etc
%   \end{itemize}
%   \item ``, '', `, and ': Quotes are like parenthesis, there are
%   opening quotes and closing quotes that are distinct. Don't use |"|
%   unless it is in a code sample.
%   \begin{list}{}
%      \item Opening single quote \fbox{`} is produced in \LaTeXe\ with a
%      single backtick (most probably the keyboard key to the left of
%      the digit 1.)
%      \item Closing single quote \fbox{'} is produced in \LaTeXe\ with a
%      single quote (most probably the keyboard key to the left of
%      the enter key.)
%      \item Opening double quotes \fbox{``} is produced in \LaTeXe\ with two
%      consecutive backticks.
%      \item Closing double quotes \fbox{''} is produced in \LaTeXe\ with two
%      single quotes.
%      \item Raw double quote character \\fbox{|"|} is produced in
%      \LaTeXe\ inside of verbatim environments by the usual double
%      quote key stroke.
%   \end{list}
%   \item When possible, compose you sentences and statements using
%   positive, rather than negative, tense.
%   \begin{itemize}
%     \item[\ding{52}] ``The tight binding of languages to hardware
%     and operating platforms makes the problem more difficult.''
%     \item[\ding{56}] ``The tight binding of languages to hardware
%     and operating platforms never made it
%     easier.''~\cite{thesis:pchaudhari}
%   \end{itemize}
%   Basically, the fewer negations you have in a sentence the easier
%   it is for the reader to figure out what you meant.
%   \item Similarly, mathematical theorems should be stated as a
%   concise, provably false statement.
%   \begin{itemize}
%     \item\ding{52} The lower bound for $\mu$ is $\frac{sqrt{6}}{2}$.
%     \item\ding{56} It can be show that the lower bound for $\mu$ is
%     $\frac{sqrt{6}}{2}$.
%   \end{itemize}
% \end{itemize}
%
% \section{Reference guide}
% \label{sec:reference}
%
% The reference guide section explains, in detail, all of the various
% document options and macors that a thesis author may wish to use
%
% \subsection{How to read the reference}
%
% Commands, keys and environments are presented as follows.
%
% \begin{syntax}
% \item[0.1,default,hints] \texttt{command}, \texttt{environment} or
%       \indexbasic{key} with \meta{parameters}
%
%       This field contains the explanation; here we describe the
%       other fields.
%
%       If present, the label in the left margin provides extra
%       information: `\textit{addon}' indicates additionally
%       introduced functionality, `\textit{changed}' a modified key,
%       `\textit{data}' a command just containing data (which is
%       therefore adjustable via |\renewcommand|), and so on. Some
%       keys and functionality are `\emph{bug}'-marked or with a
%       \dag-sign. These features might change in future or could be
%       removed, so use them with care.
%
%       If there is verbatim text touching the right margin, it is the
%       predefined value.
%
%       The label in the right margin is the current version number
%       and marks newly introduced features.
% \end{syntax}
% Regarding the parameters, please keep in mind the following:
% \begin{enumerate}
% \item A list always means a comma separated list. You must put
% braces around such a list. Otherwise you'll get in trouble with the
% \texttt{keyval} package; it complains about an undefined key.
% \item You must put parameter braces around the whole value of a key
% if you use an \oarg{optional argument} of a key inside an optional
% \oarg{key=value list}:
% |\begin{lstlisting}[caption=|{|{|}|[one]two|{|}|}|]|.
% \item Brackets `|[ ]|' usually enclose optional arguments and must
% be typed in verbatim. Normal brackets `[ ]' always indicate an
% optional argument and must not be typed in. Thus |[*]| must be typed
% in exactly as is, but [|*|] just gets |*| if you use this argument.
% \item A vertical rule indicates an alternative, e.g.~^^A
% \meta{true$\vert$false} allows either \texttt{true} or
% \texttt{false} as arguments.
% \item If you want to enter one of the special characters |{}#%\|,
% this character must be escaped with a backslash. This means that you
% must write |\}| for the single character `right brace'---but of
% course not for the closing parameter character.
% \end{enumerate}
%
% \subsection{\textsf{CSUNthesis} class file usage}
%
%   The usage of the \textsf{CSUNthesis} class file is broken into
%   three distinct sections.
%   \begin{enumerate}
%      \item Document options. These are values that can be passed as
%      options to
%      the |\documentclass[|$\ldots$|]{CSUNthesis}|
%      line. These options control very high level document formats
%      such as typeface size or draft/final options.
%      \item Required preamble commands. These commands must be
%      present for the thesis source file to compile.
%      \item Optional preamble commands. These commands provide
%      additional, or optional, features.
%   \end{enumerate}
%
%  The preamble section of the source files is the top of the source
%  file prior to the |\begin{document}| command. This is where
%  all of the class commands (except for \verb|\references|) are
%  given. Order of the commands in the
%  preamble generally does not matter. Beyond the beginning of the
%  document an author simply uses any \TeX\ or \LaTeX\ commands he
%  wishes to use in order to produce the desired publication.
%
%  The author uses the \verb|\references| command at the point in his
%  document where the bibiliography is expected to appear. This is
%  always just before the first appendix, or at the very end of the
%  document if no appendicies are included.
%
% \subsection{Document class options}
%
%   The first non-comment line of a \LaTeX document is always the
%   |\documentclass| command. Its full syntax is\begin{quote}
%   \verb|\documentclass[|$opt_1,opt_2,opt_3,$\ldots$,opt_n$\verb|]{CSUNthesis}|
%   \end{quote}
%   The bracketed part is optional and allows an author to change
%   certain document options such as default point size. It also allows
%   an author to change certain aspects of his thesis. The following
%   options are available:
%
% \begin{syntax}
%   \item[0.1,final]
%   $\langle$\indexbasic{draft}$\vert$\indexbasic{draftcls}$\vert$\indexbasic{final}$\rangle$
%
%      Publications go through numerous proof-reading and rewriting
%      iterations before they are ready to be published. During
%      proof-reading it is convenient to have a larger baseline
%      skip (inter-line spacing) so that corrections can be marked
%      manually with red ink easily.
%
%      Specifying \texttt{final} (which is the default) will format the
%      document with normal baseline skip (single spacing).
%
%      Specifying \texttt{draft} or \texttt{draftcls} will increase
%      the baseline skip to $150\%$ allowing for easy insertion of
%      manual corrections. \texttt{draft} affects the
%      underlying \emph{report} document style while \texttt{draftcls}
%      will only affect items defined by the \textsf{CSUNthesis}
%      class. Most authors will only want to use \texttt{draftcls}.
% \end{syntax}
%
% \begin{syntax}
%   \item[0.1,12pt]
%   $\langle$\indexbasic{10pt}$\vert$\indexbasic{11pt}$\vert$\indexbasic{12pt}$\rangle$
%
%   Selects the normal typeface size for the document.
%
%   Most authors at CSU Northridge will want the default of 12pt; but
%   10pt is acceptable too.
% \end{syntax}
%   
% \begin{syntax}
% \item[0.1,thesis]
% $\langle$\indexbasic{thesis}$\vert$\indexbasic{project}$\vert$\indexbasic{abstract}$\rangle$
%
% Specify one of these options to select the type of thesis. This will
% change the title page to use the appropriate wording.
%
% \begin{list}{}{}
% \item[\texttt{thesis}:] Works of an intellectual nature that
% develops and investigates new scholarly information.
% \item[\texttt{project}:] Works of a project nature where the effort
% results in a functional product based on known information.
% \item[\texttt{abstract}:] Works of a nature that defy textual
% publication such as theatre performances or artistic exhibits.
% \end{list}
% \end{syntax}
%
% \begin{syntax}
% \item[0.1,nolot]
%      $\langle$\indexbasic{lot}$\vert$\indexbasic{nolot}$\rangle$
%
% Publications that make heavy use of tabular materials and references
% to them should have a \emph{list of tables} in the front
% matter. Specifying \texttt{lot} will cause such a list to be built
% and inserted after the table of contents.
% \end{syntax}
%
% \begin{syntax}
% \item[0.1,nolof]
%      $\langle$\indexbasic{lof}$\vert$\indexbasic{nolof}$\rangle$
%
% Publications that make heavy use of figures and references to them
% should have a \emph{list of figures} in the front
% matter. Specifying \texttt{lof} will cause such a list to be built
% and inserted after the table of contents and the list of
% tables.
% \end{syntax}
%
% \begin{syntax}
% \item[0.8,nolol]
%      $\langle$\indexbasic{lol}$\vert$\indexbasic{nolol}$\rangle$
%
% Publications that make heavy use of source code listings and
% references to them should have a \emph{list of listings} in the front
% matter. Specifying \texttt{lol} will cause such a list to be built
% and inserted after the table of contents, the list of tables and
% list of figures.
%
% Authors that specify the |lol| option \textbf{must} include
% |\usepackage{listings}| in their thesis so that the
% listings package~\cite{url:listings}\indexbasic{listings.sty} will be
% available when the list of listings is built.
% \end{syntax}
%
% \begin{syntax}
% \item[0.1]
%      $\langle$\indexbasic{proposal}$\rangle$
%
% The presence of the |proposal| document option causes a radical
% change in the format of the thesis. The result is a much simpler and
% informal format suitable for presenting a proposal to the graduate
% coordinator of Computer Science.
%
% Front matter is no longer produced, section numbering is changed and
% a proposal title page with signature entries suitable for a proposal
% is created.
%
% Applicability of the proposal format for other departments
% should be checked with the graduate coordinator of the
% intended department before proceeding.
%
% \end{syntax}
%
% \subsection{Required preamble commands}
%
% The following commands must be present and appear prior to
% |\begin{document}|. The order that they appear does not matter.
%
% \subsubsection{Items mandatory for all documents}
% \begin{syntax}
% \item[0.1,,required]\indexcommand{Required}{\author}|{|\meta{author--name}|}|
%
% Specifies the primary author's name. This affects the authorship
% material present on the title, copyright, signature and abstract pages.
%
% (If necessary, see \indexslash{\collaboration} for multiple authors.)
% \end{syntax}
%
% \begin{syntax}
% \item[0.1,,required]\indexcommand{Required}{\committee}|{|\meta{chair
% member}|}{|\meta{2$^{nd}$ member}|}{|\meta{$3^{rd}$ member}|}|
%
% the |\committee| command is used to specify the proper names of the
% committee members. The three names given to the command will appear
% on the signature in the proper order. The first name is that of the
% committee chair. the suffix ``, chair'' will be automatically
% appended to the name given for the signature page.
% \end{syntax}
%
% \begin{syntax}
% \item[0.1,,required]\indexcommand{Required}{\title}|{|\meta{title of work}|}|
%
% Specifies the title of the work. This affects the signature and
% abstract pages.
% \end{syntax}
%
% \subsubsection{Additional mandatory item only for proposals}
% \begin{syntax}
% \item[0.1,,(proposal) required]
%      \indexcommand{Required}{\coordinator}|{|\meta{coordinator--name}|}|
%
% Computer Science graduate students must first submit a proposal of
% their thesis topic. The proposal requires the graduate coordinator's
% signature. The |\coordinator| command serves to specify the graduate
% coordinator's name when producing a proposal.
%
% \end{syntax}
%
% \subsubsection{Addtional mandatory items only for theses}
% \begin{syntax}
% \item[0.1,,(thesis) required]\indexcommand{Required}{\submitted}|{|\meta{month}|}{|\meta{year}|}|
%
% Specifies the month and year that the work is submitted to the
% graduate evaluator.
% \end{syntax}
%
% \begin{syntax}
% \item[0.1,,(thesis) required]\indexcommand{Required}{\abstract}|{|\meta{abstract material}|}|
%
% Dictates the body of the abstract on the abstract page. The header
% of the abstract page is created automatically and the material
% specified with the |\abstract| command appears beneath the generated
% header.
% \end{syntax}
%
% \subsection{Optional preamble commands}
%
% \begin{syntax}
% \item[0.1,,optional]
%      \indexcommand{Optional}{\degree}|{|\meta{degree}|}{|\meta{department}|}|
%
% The |\degree| command provides the author with the ability to
% specify a degree objective and department of discipline other than
% ``Master of Science'' in ``Computer Science'', which is the default
% since the \textsf{CSUNthesis} class was originally written for
% Computer Science graduate students.
% \end{syntax}
%
% \begin{syntax}
% \item[0.1,,optional]
%    \indexcommand{Optional}{\defense}|{|\meta{dayname}|}{|\meta{date}|}{|\relax
%    \meta{time}|}{|\meta{location}|}|
%
%    Students must submit an announcement page to the department of
%    Computer Science prior to their defense. The page is posted in
%    display cabinets to inform colleagues of the event time and
%    place.
%
%    The presence of the |\defense| command causes an appropriate
%    announcement page to appear prior to the title page. To suppress the
%    announcement page simply comment out the |\defense| command or
%    omit it entirely.
%
%    \begin{list}{}{}
%    \item[\emph{dayname}:] The day of the week the defense is being
%    held.\\\emph{$\langle$Monday$\vert$Tuesday$\vert$Wednesday
%    $\vert$Thursday$\vert$Friday$\rangle$}.
%    \item[\emph{date}:] Formal date the defense is being held. Such
%    as \emph{December $3^\textrm{rd}$}.
%    \item[\emph{time}:] Time the defense is being held. Such
%    as \emph{3:30PM}.
%    \item[\emph{location}:] Location the defense is being held. Such
%    as \emph{EA1440}.
%    \end{list}
% \end{syntax}
%
% \begin{syntax}
% \item[0.1,,optional]\indexcommand{Optional}{\copyrightyear}|{|\meta{year}|}|
%
% It is acceptable to copyright the work produced. Using the
% \indexslash{\copyrightyear} command with a year will cause a copyright front
% matter page to be inserted after the title page and prior to the
% signature page.
% \end{syntax}
%
% \begin{syntax}
% \item[0.1,,optional]\indexcommand{Optional}{\dedication}|{|\meta{text}|}|
%
% Authors may wish to dedicate the published work to special
% people. Including a |\dedication| command will cause the text
% argument to be formatted as a dedication page and included in the
% front matter.
% \end{syntax}
%
% \begin{syntax}
% \item[0.1,,optional]\indexcommand{Optional}{\preface}|{|\meta{text}|}|
%
% Authors may wish to include a preface to the main body of the
% work. Including a |\preface| command will cause the text
% argument to be formatted as a preface and included in the
% front matter.
% \end{syntax}
%
% \begin{syntax}
% \item[0.1,,optional]\indexcommand{Optional}{\acknowledgement}|{|\meta{text}|}|
%
% Authors may wish to acknowledge additional people for supporting
% their work. Including an |\acknowledgement| command will cause the
% text argument to be formatted as an acknowledgement page and
% included in the front matter.
% \end{syntax}
%
% \begin{syntax}
% \item[0.1,,optional]\indexcommand{Optional}{\contact}|{|\meta{contact--info}|}|
%
% On the proposal title page it is polite to provide contact
% information such as an e-mail address so that committee members can
% contact the author if they need to.
%
% The |\contact| command is used to specify the contact content for
% the proposal title page. When printing a formal thesis the contact
% information, if specified, is suppressed from the title page.
% \end{syntax}
%
% \begin{syntax}
% \item[0.1,,optional]
%      \indexcommand{Optional}{\collaboration}|{|\meta{coauthor--name}|}|
%
%   On rare occasions a thesis is produced through collaboration. To
%   identify a second author on the title page the command
%   |\collaboration| can be used to specify the second author.
% \end{syntax}
%
% \begin{syntax}
% \item[0.1,,necessary]\indexcommand{Optional}{\references}|{|\meta{style}|}{|\meta{file}|}|
%
% A bibliography is generated at the point in the document
% where the author issues the |\references| macro. The first argument,
% \emph{style}, is the style to be passed internally to \LaTeX's
% |\bibliographicstyle| command. The second argument, \emph{file}, is
% the name of the author's bibliography database file without the
% ``\texttt{.bib}'' suffix. The file argument is passed internally to
% \LaTeX's |\bibliography| command.
% This macro also causes an entry to be inserted into the table of
% contents as to the presence and location of the references section.
% \end{syntax}
%
% \StopEventually{
%   \bibliographystyle{plain}
%   \bibliography{jwcsun}
%   \PrintChanges\PrintIndex
% }
%
% \section{Implementation}
% \label{sec:implementation}
% The material in Section~\ref{sec:implementation},
% Pages~\pageref{sec:implementation}--\pageref{sec:implementation_end}
% is the actual \LaTeXe\ implementation of the
% \textsf{CSUNthesis} class. It is present for completeness and to
% encourage continued development. It is not necessary for thesis
% authors to review or understand such material.
%
% In fact for beginners it can be overwhelming and really scary.
% An interesting fact is that this documentation and the
% \textsf{CSUNthesis} class are maintained as the exact same file
% (\texttt{CSUNthesis.dtx}). Compiling the \texttt{.dtx} file with latex
% produces the \texttt{.cls} (class) and the \texttt{.dvi}
% (documentation) file. So it is impossible to omit the scary, but
% very useful, information presented here in
% Section~\ref{sec:implementation} as doing so would remove it from
% the class file as well!
%
% The other advantage to this section is that it is the actual
% \LaTeXe\ code that is present in the class file. So adventurous
% authors can learn how the \textsf{CSUNthesis} class file was
% engineered and can even change it to suit their specific
% needs. (The original author of the \textsf{CSUNthesis} class
% \textbf{highly} recommends that changes be suggested to him or
% back-ported so that future students can benefit from continued
% refinement and development.
%
% \begin{category}{Identification}
% Print information during compilation to indicate what version of the
% \texttt{CSUNthesis.cls} file was found and loaded.
%    \begin{macrocode}
\typeout{-- CSUN Thesis style} \typeout{-- Author: Wiegley, Jeff,
jeffw@csun.edu}
%    \end{macrocode}
%
% Request that mistakes be reported so they may be corrected for all
% students.
%    \begin{macrocode}
\typeout{--   Please contact the author to have any mistakes corrected.}
%    \end{macrocode}
% \end{category}
%
% \begin{category}{Typeface}
% \changes{1.3}{19 Apr 2005}
%   {Changed default typeface family to Times (Blech!)}
% \changes{2.1}{01 Dec 2005}
%   {removed changes to math typeface. math is now cmr again to allow bold.}
% Graduate evaluator demands Times New Roman. They shouldn't; but they
% do. To conform to this we use the \texttt{times} package and we
% execute a few work arounds to get math mode looking similar.
% (It may make |\texttt{}| look strange because that is an entirely
% typeface family than Times New
%    \begin{macrocode}
\usepackage{times}
% \iffalse meta-comment
% Do not use mathptm because:
%  - \boldmath produces non-italic characters
%  - \boldmath seems ultra-fragile, usage in redifining sectioning
%    macros produces non-bold behavior.
% \usepackage{mathptm}
% Do not use mathptm because:
%  - mathtime has no bold support
% We should use mathtime with mtbold option BUT it doesn't map
% characters properly and everything is just plain screw up.
% one area that latex *sucks* at is providing font options.
% And YES I do know about installing postscript fonts and all that jazz but:
%   A) I shouldn't have to know this crap to select a font.
%   B) Not all distributions have a decent set of fonts pre-installed
%   C) Fonts aren't free.
%   D) There's no way a new student/author will be able to figure it out
% \usepackage[mtbold]{mathtime}
% \fi
%    \end{macrocode}
% \end{category}
% \begin{category}{Custom lengths}
% The title on the title page is typeset in a minipage of width
% |\titlewidth|. This prevents ugly, wide titles from being produced.
%    \begin{macrocode}
\newlength{\titlewidth}
\setlength{\titlewidth}{4.5in}
%    \end{macrocode}
% Some of the front matter pages have their material start further
% down the page. |\frontmattertopmargin| controls the amount of
% additional space to add to the normal top margin for these pages.
%    \begin{macrocode}
\newlength{\frontmattertopmargin}
\setlength{\frontmattertopmargin}{1.0in}
%    \end{macrocode}
% \end{category}

% \begin{category}{Conditionals}
% Several conditionals are need to keep track of format specifications
% and control the final output.
%
% |\ifproposal| controls the formality of the document.
%    \begin{macrocode}
\newif\ifproposal\proposalfalse
%    \end{macrocode}
% one of |\ifthesis|, |\ifproject|, or |\ifabstract| gets set to true
% and controls wording in the front matter based on the presence of
% the document options \indexbasic{thesis}, \indexbasic{project} or
% \indexbasic{abstract}.
%    \begin{macrocode}
\newif\ifthesis\thesisfalse
\newif\ifabstract\abstractfalse
\newif\ifproject\projectfalse
%    \end{macrocode}
% |\iflof|, |\iflot| and |\iflol| control whether or not the front
% matter includes a list of figures, list of tables and a list of
% listings, respectively. All default to false. If the author supplies
% the |lol| option then the |listings| package will automatically be
% included.
%    \begin{macrocode}
\newif\iflof\loffalse
\newif\iflot\lotfalse
\newif\iflol\lotfalse
%    \end{macrocode}
% |\ifdraft| controls whether the baseline skip is $100\%$ (false) or
% $150\%$ (true) of normal.
%    \begin{macrocode}
\newif\ifdraft\draftfalse
%    \end{macrocode}
% |\ifsizespec| maintains whether or not a typesize option has been
% specified.
%    \begin{macrocode}
\newif\ifsizespec\sizespecfalse
%    \end{macrocode}
% |\ifmadebib| maintains whether or not the bibliography chapter has
% already been produced.
%    \begin{macrocode}
\newif\ifmadebib\madebibfalse
%    \end{macrocode}
% \end{category}
%
% \begin{category}{Illegal options}
% This class bases its format on the |report| style. Several options
% available to the |report| style make no sense in the context of a
% CSU Northridge thesis. The following declarations cause class errors
% if such options are present. This action effectively disables the
% possibility of successfully using any of the listed options.
%
% \begin{option}{twocolumn}{Document options\levelchar Illegal}
% Two column format is standard in many conference proceedings but are
% not allowed by the CSU Northridge graduate evaluator.
%    \begin{macrocode}
\DeclareOption{twocolumn}{
  \OptionNotUsed
  \ClassError{CSUNthesis}{only single column documents allowed}{}
}
%    \end{macrocode}
% \end{option}
% \begin{option}{twosided}{Document options\levelchar Illegal}
% Thesis need to be submitted in a format suitable for photo
% typesetting. This requires that the document be single sided.
%    \begin{macrocode}
\DeclareOption{twosided}{
  \OptionNotUsed
  \ClassError{CSUNthesis}{only single sided publications allowed}{}
}
%    \end{macrocode}
% \end{option}
% \begin{option}{8pt}{Document options\levelchar Illegal}
% Thesis are allowed to be 10 or 12 points. Typeface sizes of 8 points are
% also too small to meet the requirements of the graduate evaluator.
%    \begin{macrocode}
\DeclareOption{8pt}{
  \OptionNotUsed
  \ClassError{CSUNthesis}{only 10pt or 12pt typeface allowed}{}
}
%    \end{macrocode}
% \end{option}
% \begin{option}{9pt}{Document options\levelchar Illegal}
% Thesis are allowed to be 10 or 12 points. Typeface sizes of 9 points are
% too small to meet the requirements of the graduate evaluator.
%    \begin{macrocode}
\DeclareOption{9pt}{
  \OptionNotUsed
  \ClassError{CSUNthesis}{only 10pt or 12pt typeface allowed}{}
}
%    \end{macrocode}
% \end{option}
% \begin{option}{10pt}{Document options\levelchar Legal}
% Specifies that the thesis should be typeset in 10 point typeface family.
%    \begin{macrocode}
\DeclareOption{10pt}{
  \sizespectrue
  \PassOptionsToClass{\CurrentOption}{report}  
}
%    \end{macrocode}
% \end{option}
% \begin{option}{11pt}{Document options\levelchar Legal}
% Specifies that the thesis should be typeset in 11 point typeface family.
%    \begin{macrocode}
\DeclareOption{11pt}{
  \sizespectrue
  \PassOptionsToClass{\CurrentOption}{report}  
}
%    \end{macrocode}
% \end{option}
% \begin{option}{12pt}{Document options\levelchar Legal}
% Specifies that the thesis should be typeset in 12 point typeface family.
%    \begin{macrocode}
\DeclareOption{12pt}{
  \sizespectrue
  \PassOptionsToClass{\CurrentOption}{report}  
}
%    \end{macrocode}
% \end{option}
% \begin{option}{a4paper}{Document options\levelchar Illegal}
% American institutions all use \emph{letter} sized paper
% ($8.5\times11$ inches) and not \emph{a4} as is used in Europe.
%    \begin{macrocode}
\DeclareOption{a4paper}{
  \OptionNotUsed
  \ClassWarning{CSUNthesis}{CSU requires letter sized paper, a4paper
    ignored}{}
}
%    \end{macrocode}
% \end{option}
% \end{category}
% \begin{category}{Thesis types}
% Masters thesis come in three varieties. Certain wording in the front
% matter needs to be changed accordingly. The \indexbasic{thesis},
% \indexbasic{project} and \indexbasic{abstract} document options set the
% conditionals that control such wording.
% \begin{option}{thesis}{Document options\levelchar Legal}
% Selects the wording to be ``graduate thesis'', causes an error if a
% different format has already been specified
%    \begin{macrocode}
\DeclareOption{thesis}{
  \ifproject
  \ClassError{CSUNthesis}{only one thesis type may be specified}{}
  \fi
  \ifabstract
  \ClassError{CSUNthesis}{only one thesis type may be specified}{}
  \fi
  \thesistrue
}
%    \end{macrocode}
% \end{option}
% \begin{option}{project}{Document options\levelchar Legal}
% Selects the wording to be ``graduate project'', causes an error if a
% different format has already been specified
%    \begin{macrocode}
\DeclareOption{project}{
  \ifthesis
  \ClassError{CSUNthesis}{only one thesis type may be specified}{}
  \fi
  \ifabstract
  \ClassError{CSUNthesis}{only one thesis type may be specified}{}
  \fi
  \projecttrue
}
%    \end{macrocode}
% \end{option}
% \begin{option}{abstract}{Document options\levelchar Legal}
% Selects the wording to be ``graduate abstract'', causes an error if a
% different format has already been specified
%    \begin{macrocode}
\DeclareOption{abstract}{
  \ifthesis
  \ClassError{CSUNthesis}{only one thesis type may be specified}{}
  \fi
  \ifproject
  \ClassError{CSUNthesis}{only one thesis type may be specified}{}
  \fi
  \abstracttrue
}
%    \end{macrocode}
% \end{option}
% \end{category}

% \begin{category}{Switch options}
% The front matter may contain lists of materials other than the
% \emph{table of contents}. The \textsf{CSUNthesis} class provides
% a number of document options that control the presence or absence of
% these front matter pages.
% \begin{option}{lof}{Switch options}
% |lof| causes a \emph{list of figures}\indexbasic{list of figures} to be
% included in the front matter. 
%    \begin{macrocode}
\DeclareOption{lof}{\loftrue}
%    \end{macrocode}
% \end{option}
% \begin{option}{nolof}{Switch options}
% |nolof| prevents the \emph{list of figures}\indexbasic{list of figures}
% from being included in the front matter. (This is the default if
% neither |lof| or |nolof| is specified.
%    \begin{macrocode}
\DeclareOption{nolof}{\loftrue}
%    \end{macrocode}
% \end{option}
% \begin{option}{lol}{Switch options}
% |lol| causes a \emph{list of listings}\indexbasic{list of listings} to be
% included in the front matter. 
%    \begin{macrocode}
\DeclareOption{lol}{\loltrue}
%    \end{macrocode}
% \end{option}
% \begin{option}{nolol}{Switch options}
% |nolol| prevents the \emph{list of listings}\indexbasic{list of listings}
% from being included in the front matter. (This is the default if
% neither |lol| or |nolol| is specified.
%    \begin{macrocode}
\DeclareOption{nolol}{\loltrue}
%    \end{macrocode}
% \end{option}
% \begin{option}{lot}{Switch options}
% |lot| causes a \emph{list of tables}\indexbasic{list of tables} to be
% included in the front matter. 
%    \begin{macrocode}
\DeclareOption{lot}{\lottrue}
%    \end{macrocode}
% \end{option}
% \begin{option}{nolot}{Switch options}
% |nolot| prevents the \emph{list of tables}\indexbasic{list of tables}
% from being included in the front matter. (This is the default if
% neither |lot| or |nolot| is specified.
%    \begin{macrocode}
\DeclareOption{nolot}{\lottrue}
%    \end{macrocode}
% \end{option}
% \begin{option}{draft}{Switch options}
% The \texttt{draft} and \texttt{draftcls} options cause the baseline
% skip to be increased to $150\%$ of normal. This facilitates
% proof-reading and corrections as it provides space for correction
% notations and makes such corrections easy to spot. They should not
% be used for the final publication.
%    \begin{macrocode}
\DeclareOption{draft}{\drafttrue\AtBeginDocument{\onehalfspacing}}
%    \end{macrocode}
% \end{option}
% \begin{option}{draftcls}{Switch options}
%   The \texttt{draft} and \texttt{draftcls} options are passed to the
%   underlying \texttt{report} document style. \texttt{report}
%   recognizes \texttt{draft} and will change many behaviors including
%   spacing in the table of contents and in captions. This is usually
%   not desirable.  Most thesis authors will want to use the
%   \texttt{draftcls} option instead. \texttt{draftcls} is not
%   recognized by \texttt{report} and therefore only affects the
%   format of the \textsf{CSUNthesis} features.
%    \begin{macrocode}
\DeclareOption{draftcls}{\drafttrue\AtBeginDocument{\onehalfspacing}}
%    \end{macrocode}
% \end{option}
% \begin{option}{final}{Switch options}
%   For final publication authors should either specify the document
%   option \texttt{final} or specify no draft/final option as
%   \texttt{final} is the default. This produces a document with a
%   normal baseline skip.
%    \begin{macrocode}
\DeclareOption{final}{\draftfalse\AtBeginDocument{\singlespacing}}
%    \end{macrocode}
% \end{option}
% \begin{option}{proposal}{Switch options}
%   A simple switch that changes the format to a much more informal
%   style that is adequate for producing a thesis proposal in the
%   department of computer science.
%    \begin{macrocode}
\DeclareOption{proposal}{\proposaltrue}
%    \end{macrocode}
% \end{option}
% \end{category}

% \begin{category}{Base Style}
%   The \textsf{CSUNthesis} class file is based on the basic
%   \texttt{report} style of \LaTeX. A boolean is used to determine
%   if the author already specified a typesize that was, in turn,
%   scheduled for forwarding to the \emph{report} class. If no
%   specific size option was provided then a default of 12pt is passed
%   on to the \emph{report} class.
%    \begin{macrocode}
\DeclareOption*{\PassOptionsToClass{\CurrentOption}{report}}
\ExecuteOptions{final}
\ProcessOptions\relax
\ifsizespec\relax\else\PassOptionsToClass{12pt}{report}\fi
\LoadClass{report}
%    \end{macrocode}
% \end{category}
% \begin{category}{Change tracking}
% \changes{1.5}{1 May 2005}
%  {Added rudimentary Change tracking capabilities}
% \LaTeXe\ unfortunately does not do change tracking (also known as
% ``red-lining'' very well. To provide a rudimentary feature the
% |\edit{}|, |\delete{}| and |\add{}| macros are defined here.
%
% The change tracking macros change the color of text. This requires
% the color pacage
%    \begin{macrocode}
%%%%%\usepackage[usenames]{color}
%    \end{macrocode}
% The change tracking macros also want to strike-through text. This
% feature requires the ulem package and that normal emphasis style be
% restored.
%    \begin{macrocode}
\usepackage{ulem}
\normalem
%    \end{macrocode}
% \begin{macro}{\udot}
% \changes{1.5}{1 May 2005}
%  {Added dotted and dashed underlining methods}
% Because must documents are destined to be printed on a monochrome
% device the color chnages that highlight changes may be lost during
% printing. To prevent loss of context all changes are also indicated
% by some sort of non-color visual. Deletions are struck through,
% changes are underlined nd notes are underlined with a wavy
% line. Something is also needed for additions. The ulem package
% doesn't have many predefined underline type but a couple are easy to
% implement. We define |\udot{}| for use with |\add| which produces an
% unobtrusive dotted underline style. A dashed underline style is also
% defined.
%    \begin{macrocode}
\newcommand{\udot}{\bgroup \markoverwith{\lower .4ex\hbox{.}}\ULon}
\newcommand{\udash}{\bgroup \markoverwith{\lower .8ex\hbox{-}}\ULon}
%    \end{macrocode}
% \end{macro}
% \begin{macro}{\marginreason}
% \changes{1.9}{20 Oct 2005}
%  {change tracking: empty reason suppresses marginpar}
% \LaTeXe\ can only handle 18 floats per page. This causes problems
% with change tracking because margin paragraphs (notes) are
% implemented as floats. Therefore if every change produces a margin
% note then you are limited to 18 changes per page. This is
% unacceptable because there are frequently more changes required per
% page and some changes have an obvious reason such as punctuation,
% spacing and capitalization. So we implement |\marginreason{}{}|
% which takes two arguments, The first being the color to format the
% reason and the second being the reason. If the reason is left empty
% (not just blank... empty |{}|) then the actual call to
% |\marginpar{}| is suppressed.
%
% This macro is not perfect!! There are instances where it will fail.
% This code was adapted from \url{http://groups.google.com/group/comp.text.tex/browse_frm/thread/fdf45ac7b2eeff5f/2fff0f06937be0e7}.
%    \begin{macrocode}
\newcommand{\marginreason}[2]{%
\def\thereason{#2}%
\ifx\@empty\thereason\relax\else\marginpar{\color{#1}#2}\fi%
}
%    \end{macrocode}
% \end{macro}
% \begin{macro}{\edit}
%   Text inside the |\edit macro| should be underlined and have it's color changed to red to indicate that something is wrong and needs to be corrected. The second argument produces a marginpar that explains the needed change.
%    \begin{macrocode}
\newcommand\edit[2]{%
  \typeout{WARNING: edit change tracking macro exists in source}%
  {\color{Red}\uline{#1}}\marginreason{Red}{#2}}%
%    \end{macrocode}
% \end{macro}
% \begin{macro}{\delete}
%   Text inside the |\delete macro| is struck-through and has it's
%  color changed to red to indicate that something should be deleted.
%    \begin{macrocode}
\newcommand\delete[2]{%
  \typeout{WARNING: delete change tracking macro exists in source}%
  {\color{Red}\sout{#1}}\marginreason{Red}{#2}}%
%    \end{macrocode}
% \end{macro}
% \begin{macro}{\add}
%   Text inside the |\add macro| is inserted into the document
%  underlined and red. The second argument is formatted in a margin
%  paragraph to explain the change.
%    \begin{macrocode}
\newcommand\add[2]{%
  \typeout{WARNING: add change tracking macro exists in source}%
  {\color{Red}\udot{#1}}\marginreason{Red}{#2}}%
%    \end{macrocode}
% \end{macro}
% \begin{macro}{\replace}
%   Text inside the |\replace macro| is marked for deletion. The
%  second macro specifies what to replace with and the third is
%  formatted as the margin paragraph. The formatting is identical to delete
%  followed by add
%%    \begin{macrocode}
\newcommand\replace[3]{%
  \typeout{WARNING: replace change tracking macro exists in source}%
  {\color{Red}\sout{#1}\udot{#2}}\marginreason{Red}{#3}}%
%    \end{macrocode}
% \end{macro}
% \begin{macro}{\note}
%   Text inside the |\note macro| inserts text that is descriptive but
%  not intended as part of the final document. Wavy underline is used
%  to help provide the note context. All notes should be removed prior
%  to final publication. No reason argument is provided. 
%%    \begin{macrocode}
\newcommand\note[1]{%
  \typeout{WARNING: replace change tracking macro exists in source}%
  {\color{Red}\uwave{#1}}}%
%    \end{macrocode}
% \end{macro}
% \end{category}

% Although, by default, \LaTeX\ has a very professional appearance
% the CSU Northridge graduate evaluator dictates a more traditional
% ``typewriter'' appearance. It is therefore necessary to alter many
% of the basic \LaTeX\ commands. The required commands are
% re-implemented here.
%
% \begin{macro}{\contentsname}
%   The \LaTeX\ default table of contents is labeled ``Contents'' the
%   graduate evaluator dictates a heading of ``Table of
%   Contents''. This change is minor and is implemented by simply
%   redefining the default string.
%    \begin{macrocode}
\renewcommand\contentsname{\normalfont Table of Contents}
%    \end{macrocode}
% \end{macro}
% \begin{macro}{\bibname}
% \changes{1.2}{18 Apr 2005}
%   {Changed bibliography title to ``References''}
%   The \LaTeX\ default table of contents is labeled ``Contents'' the
%   graduate evaluator dictates a heading of ``Table of
%   Contents''. This change is minor and is implemented by simply
%   redefining the default string.
%    \begin{macrocode}
\renewcommand\bibname{References}
%    \end{macrocode}
% \end{macro}
% \begin{macro}{\@makechapterhead}
%   All modern, professional publications present chapter headings
%   left justified, in a larger typeface and typeset with a larger top
%   margin. This is contrary to the graduate evaluator's
%   specifications. The |\@makechapterhead| is redefined for centering
%   and normal to comply.
% \changes{2.0}{30 Nov 2005}
%   {added bold math to chapter headings}
%    \begin{macrocode}
\def\@makechapterhead#1{%
  {\parindent \z@ \centering \normalfont
    \ifnum \c@secnumdepth >\m@ne
        \bfseries\boldmath \@chapapp\space \thechapter
        \par\nobreak
    \fi
    \interlinepenalty\@M
    \bfseries\boldmath #1\par\nobreak
    \vskip 20\p@
  }}
%    \end{macrocode}
% \end{macro}

% \begin{macro}{\@makeschapterhead}
%   the |\begin{chapter*}| must also be changed in case authors want
%   a chapter that is not listed in the table of contents.
%    \begin{macrocode}
\def\@makeschapterhead#1{%
  {\parindent \z@ \centering
    \normalfont
    \interlinepenalty\@M
    \bfseries\boldmath #1\par\nobreak
    \vskip 20\p@
  }}
%    \end{macrocode}
% \end{macro}

% \begin{macro}{\@makepchapterhead}
%   like \verb|\@makechapterhead| a correct
%   \verb|\@makepchapterhead| needs to be supplied so that preface
%   chapter heads are produced correctly.
%    \begin{macrocode}
\def\@makepchapterhead#1{%
  {\parindent \z@ \centering
    \normalfont
    \interlinepenalty\@M
    #1\par\nobreak
    \vskip 20\p@
  }}
%    \end{macrocode}
% \end{macro}

% \begin{macro}{\prefacechapter}
%   Preface chapters are not allowed to be bold. A different chapter
%   type is created to allow for preface chapters to formatted correctly.
%    \begin{macrocode}
\def\prefacechapter#1{\if@twocolumn
                   \@topnewpage[\@makepchapterhead{#1}]%
                 \else
                   \@makepchapterhead{#1}%
                   \@afterheading
                 \fi}
%    \end{macrocode}
% \end{macro}

% \begin{macro}{\listoffigures}
%   The list of figures is a preface page and all preface pages are not
%   permitted to have bold wieght text. We call a different |\chapter*|
%   type command that is identical to |\chapter*| but supresses boldface.
%    \begin{macrocode}
\renewcommand\listoffigures{%
    \if@twocolumn
      \@restonecoltrue\onecolumn
    \else
      \@restonecolfalse
    \fi
    \prefacechapter{\listfigurename}%
      \@mkboth{\MakeUppercase\listfigurename}%
              {\MakeUppercase\listfigurename}%
    \@starttoc{lof}%
    \if@restonecol\twocolumn\fi
    }
%    \end{macrocode}
% \end{macro}

% \begin{macro}{\listoffigures}
%   The list of tables is a preface page and all preface pages are not
%   permitted to have bold wieght text.
%    \begin{macrocode}
\renewcommand\listoftables{%
    \if@twocolumn
      \@restonecoltrue\onecolumn
    \else
      \@restonecolfalse
    \fi
    \prefacechapter{\listtablename}%
      \@mkboth{%
          \MakeUppercase\listtablename}%
         {\MakeUppercase\listtablename}%
    \@starttoc{lot}%
    \if@restonecol\twocolumn\fi
    }
%    \end{macrocode}
% \end{macro}

% \begin{macro}{\section}
%   All modern, professional publications emphasize sections and
%   subsections by larger typeface, weight or vertical spacing. The
%   graduate evaluator dictates $12$ point typeface throughout a
%   thesis. The section, subsection and subsubsection commands are
%   similarly redefined to comply. \danger We do not account for the starred
%   version of these sectioning commands and so they should not be
%   used in a thesis.
%
%   \changes{2.0}{30 Nov 2005}
%     {added bold math to section headings}
%   Command for sections
%    \begin{macrocode}
\renewcommand{\section}{
  \@startsection{section}{1}{0mm}{0pt}{0.0001pt}%
		{\noindent\normalfont\normalsize\bfseries\boldmath}}%
%    \end{macrocode}
% \end{macro}

% \begin{macro}{\subsection}
%   \changes{2.0}{30 Nov 2005}
%     {added bold math to subsection headings}
%   \changes{2.0}{11 Aug 2006}
%     {changed spacing before and after section headings}
%   Command for subsections
%    \begin{macrocode}
\renewcommand{\subsection}{
  \@startsection{subsection}{1}{0mm}{0pt}{0.0001pt}%
		{\noindent\normalfont\normalsize\bfseries\boldmath}}%
%    \end{macrocode}
% \end{macro}

% \begin{macro}{\subsubsection}
%   \changes{2.0}{30 Nov 2005}
%     {added bold math to subsubsection headings}
%   Command for subsubsections
%    \begin{macrocode}
\renewcommand{\subsubsection}{
  \@startsection{subsubsection}{1}{0mm}{0pt}{0.0001pt}%
		{\noindent\normalfont\normalsize\bfseries\boldmath}}%
%    \end{macrocode}
% \end{macro}

% \begin{macro}{\l@chapter}
%   Contrary to modern publishing standards the graduate evaluator
%   does not allow any typeface size or series differences in the front
%   matter. Chapter entries in the table of contest, as provided byb
%   \LaTeX, violates this. Such entries are produced by the
%   \l@chapter command. This command is redefined here to comply.
%    \begin{macrocode}
\renewcommand*\l@chapter[2]{%
  \ifnum \c@tocdepth >\m@ne%
    \addpenalty{-\@highpenalty}%
    \vskip 1.0em \@plus\p@%
    \setlength\@tempdima{1.5em}%
    \begingroup%
      \parindent \z@ \rightskip \@pnumwidth%
      \parfillskip -\@pnumwidth%
      \leavevmode%\bfseries
      \advance\leftskip\@tempdima%
      \hskip -\leftskip%
      #1\nobreak\hfil \nobreak\hb@xt@\@pnumwidth{\hss #2}\par%
      \penalty\@highpenalty%
    \endgroup%
  \fi}
%    \end{macrocode}
% \end{macro}

% \begin{macro}{\maketitle}
%   The \texttt{report} style causes a call to |\maketitle| at the
%   |\begin{document}| command. The \textsf{CSUNthesis} class
%   creates its own front matter. So |\maketitle| is trivialized to
%   prevent its invocation from producing content.
%    \begin{macrocode}
\renewcommand{\maketitle}{}
%    \end{macrocode}
% \end{macro}

% \begin{macro}{Formality}
%   Normally, sections are labeled $X.Y$. Proposals don't have chapter
%   and thus no $X$. To prevent numbering of $0.Y$ the section
%   numbering command is redefined. \danger This will not provide
%   provide proper sub- and subsub- section numbers.
%    \begin{macrocode}
\ifproposal
\renewcommand{\thesection}{\arabic{section}}
\fi
%    \end{macrocode}
% \end{macro}

% \begin{macro}{Thesis type}
% The front matter creation pages require that at least one type be
% set in order to select the proper wording. This guarantees that at
% least |\thesistrue| will be set internally in case the
% author does not specify one at all.
%    \begin{macrocode}
\ifthesis
\relax
\else
\ifabstract\relax\else\ifproject\relax\else\thesistrue\fi\fi
\fi
%    \end{macrocode}
% \end{macro}

% \begin{macro}{Margins}
%   It is generally a better idea to use the \textsf{geometry} package
%   to specify page margins and layout dimensions. The \texttt{MikTeX}
%   package does not include the \textsf{geometry} package by
%   default. The page layout is not difficult so it is done through
%   basic \TeX\ commands so that the \textsf{CSUNthesis} class is easily
%   used by \texttt{MikTeX} users (which doesn't provide the
%   \textsf{geometry} package by default).
%
%   There is no header so header dimensions are reduced to zero.
%
%   Thesis are printed single sided so the odd side and even side
%   margins are set equal.
%    \begin{macrocode}
\setlength{\headheight}{0.0in}     % results in 1.0inch
\setlength{\headsep}{0.0in}     % results in 1.0inch
\setlength{\topmargin}{0.0in}     % results in 1.0inch
\setlength{\textheight}{9.0in}
\setlength{\footskip}{0.5in}
\setlength{\oddsidemargin}{0.5in} % results in 1.5in
\setlength{\evensidemargin}{\oddsidemargin}
\setlength{\textwidth}{6.0in}
%    \end{macrocode}
% \end{macro}

% \begin{macro}{Spacing}
%   spacing is better handled through the \textsf{setspace} package
%   because it knows how to adjust spacing for nearly all typeset
%   elements including lists and quotes.
%    \begin{macrocode}
\RequirePackage{setspace}
%    \end{macrocode}
% \end{macro}

% \begin{macro}{page style}
%   Page style of plain is the default because that centers a page
%   number and nothing else in the header and footer.
%    \begin{macrocode}
\pagestyle{plain}
%    \end{macrocode}
% \end{macro}

% \begin{macro}{\@degree}
%   The type of degree being fullfilled is stored in te |\@degree|
%   macro. It is initially set to the default here but can be changed
%   by author by issuing the |\degree| command.
%    \begin{macrocode}
\newcommand{\@degree}{Master~of~Science}
%    \end{macrocode}
% \end{macro}

% \begin{macro}{\@department}
%   The department the student is seeking the degree from is stored in
%   a variable macro named |\@department|. That variable is created
%   and initialized here to the
%   default of computer science. Though other disciplines are
%   encouraged to promote the use of \LaTeX\ amongst their graduates
%   students.
%    \begin{macrocode}
\newcommand{\@department}{Computer~Science}
%    \end{macrocode}
% \end{macro}

% \begin{macro}{\degree}
%   The |\degree| command is used to change both the degree sought and
%   the issuing department. It simply overwrites the defaults stored
%   in the corresponding variable macros.
%    \begin{macrocode}
\newcommand{\degree}[2]{
  \renewcommand{\@degree}{#1}
  \renewcommand{\@department}{#2}
}
%    \end{macrocode}
% \end{macro}

% \begin{macro}{\bib}
% \changes{1.2}{18 Apr 2005}
%   {replaced by references macro}
% \end{macro}

% \begin{macro}{\references}
% \changes{1.2}{18 Apr 2005}
%   {added}
% \changes{1.6}{03 May 2005}
%   {behavior of \\references is changed. Now works at top of document only.}
%
%   The production of the bibliography must appear before any
%   appendecies and an entry in the table of contents must
%   appear for the bibliography page title ``References.''
%
%   Authors should put the |\references{}{}| macro at the
%   top of their doccument. The references will either be produced at
%   the point where the author uses the |\appendix| macro or at the
%   end of the document, which ever occurs first.
%    \begin{macrocode}
\newcommand{\references}[2]{
  \ifx\undefined\@thesisbibstyle
  \newcommand{\@thesisbibstyle}{#1}
  \else
  \renewcommand{\@thesisbibstyle}{#1}
  \fi
  \ifx\undefined\@thesisbibfile
  \newcommand{\@thesisbibfile}{#2}
  \else
  \renewcommand{\@thesisbibfile}{#2}
  \fi
}
%    \end{macrocode}
% \end{macro}

% \begin{macro}{\submitted}
%   The author must state the month and year in which the thesis is
%   submitted to the graduate evaluator. This date is used in the
%   front matter. A formal month and four digit year shall be used.
%    \begin{macrocode}
\newcommand{\submitted}[2]{
  \ifx\undefined\@submitmonth
  \newcommand{\@submitmonth}{#1}
  \else
  \renewcommand{\@submitmonth}{#1}
  \fi
  \ifx\undefined\@submityear
  \newcommand{\@submityear}{#2}
  \else
  \renewcommand{\@submityear}{#2}
  \fi
}
%    \end{macrocode}
% \end{macro}

% \begin{macro}{\defense}
%   Each author must announce the date of their defense. An
%   announcement page is built automatically if the |\defense| command
%   is present. This macro is provided here and simply defines the four
%   time and location values that are added to the committee, author and
%   abstract information to create an announcement page.
%    \begin{macrocode}
\newcommand{\defense}[4]{
  \ifx\undefined\@defenseday
  \newcommand{\@defenseday}{#1}
  \newcommand{\@defensedate}{#2}
  \newcommand{\@defensetime}{#3}
  \newcommand{\@defenselocation}{#4}
  \else
  \renewcommand{\@defenseday}{#1}
  \renewcommand{\@defensedate}{#2}
  \renewcommand{\@defensetime}{#3}
  \renewcommand{\@defenselocation}{#4}
  \fi
}
%    \end{macrocode}
% \end{macro}

% \begin{macro}{\contact}
%   |\contact| provides the command to specify the contact content to
%   place on the title page of proposals.
%    \begin{macrocode}
\newcommand{\contact}[1]{
  \ifx\undefined\@contact
  \newcommand{\@contact}{#1}
  \else
  \renewcommand{\@contact}{#1}
  \fi
}
%    \end{macrocode}
% \end{macro}

% \begin{macro}{\collaboration}
%   Provides the ability to specify a collaborative author. Leaving it
%   undefined produces single authorship front matter.
%    \begin{macrocode}
\newcommand{\collaboration}[1]{
  \ifx\undefined\@collaborator
  \newcommand{\@collaborator}{#1}
  \else
  \renewcommand{\@collaborator}{#1}
  \fi
}
%    \end{macrocode}
% \end{macro}

% \begin{macro}{\dedication}
%   Using |\dedication| specifies the content for placing on a
%   dedication page and causes the dedication page to be added to the
%   front matter.
%    \begin{macrocode}
\newcommand{\dedication}[1]{
  \ifx\undefined\@dedication
  \newcommand{\@dedication}{#1}
  \else
  \renewcommand{\@dedication}{#1}
  \fi
}
%    \end{macrocode}
% \end{macro}

% \begin{macro}{\acknowledgement}
%   Using |\acknowledgment| specifies the content for placing on a
%   acknowledgement page and causes the acknowledgement page to be
%   added to the front matter.
%    \begin{macrocode}
\newcommand{\acknowledgement}[1]{
  \ifx\undefined\@acknowledgement
  \newcommand{\@acknowledgement}{#1}
  \else
  \renewcommand{\@acknowledgement}{#1}
  \fi
}
%    \end{macrocode}
% \end{macro}

% \begin{macro}{\preface}
%   Using |\preface| specifies the content for placing on
%  preface pages and causes preface pages to be added to the
%  front matter.
%    \begin{macrocode}
\newcommand{\preface}[1]{
  \ifx\undefined\@preface
  \newcommand{\@preface}{#1}
  \else
  \renewcommand{\@preface}{#1}
  \fi
}
%    \end{macrocode}
% \end{macro}

% \begin{macro}{\abstract}
%   Using |\abstract| specifies the content for placing on the
%   abstract page. The use of this command is required since all
%   thesis must have an abstract.
%    \begin{macrocode}
\renewcommand{\abstract}[1]{
  \ifx\undefined\@abstract
  \newcommand{\@abstract}{#1}
  \else
  \renewcommand{\@abstract}{#1}
  \fi
}
%    \end{macrocode}
% \end{macro}

% \begin{macro}{\copyrightyear}
%   The page following the title page is allowed to be a copyright
%   page. If an author desires such a page then the use of the
%   |\copyrightyear| will cause such a page to be added and uses the
%   argument as the copyright year.
%    \begin{macrocode}
\newcommand{\copyrightyear}[1]{
  \ifx\undefined\@copyrightyear
  \newcommand{\@copyrightyear}{#1}
  \else
  \renewcommand{\@copyrightyear}{#1}
  \fi
}
%    \end{macrocode}
% \end{macro}

% \begin{macro}{\coordinator}
%   Provides for the graduate coordinator's name as a signature line
%   on the proposal title page.
%    \begin{macrocode}
\newcommand{\coordinator}[1]{
  \ifx\undefined\@coordinator
  \newcommand{\@coordinator}{#1}
  \else
  \renewcommand{\@coordinator}{#1}
  \fi
}
%    \end{macrocode}
% \end{macro}

% \begin{macro}{\committee}
%   The thesis is approved by a committee of three faculty
%   members. The front matter must contain a signature space with a line
%   for each member to sign. The |\committee| command provides the
%   means to specify the names of the committee members.
%    \begin{macrocode}
\newcommand{\committee}[3]{
  \ifx\undefined\@memberA
  \newcommand{\@memberA}{#2}
  \else
  \renewcommand{\@memberA}{#2}
  \fi
  \ifx\undefined\@memberB
  \newcommand{\@memberB}{#3}
  \else
  \renewcommand{\@memberB}{#3}
  \fi
  \ifx\undefined\@memberChair
  \newcommand{\@memberChair}{#1}
  \else
  \renewcommand{\@memberChair}{#1}
  \fi
}
%    \end{macrocode}
% \end{macro}

% \begin{macro}{\frontpagesetup}
% \changes{1.1}{14 Apr 2005}
%   {removed boldface from front pages}
% \changes{1.1}{14 Apr 2005}
%   {remove 12pt typeface override, frontpage header is now same as document}
% \changes{1.6}{03 May 2005}
%   {Front matter pages' top margin changed to 1 inch}
%  The dedication, acknowledgement and preface pages have a slightly
%  larger top margin. The |\frontpagesetup| sets up a generic page in the
%  front matter for use as these types of pages.
%    \begin{macrocode}
\newcommand{\frontpagesetup}[1]{
%  \vspace*{\frontmattertopmargin}
  \begin{center}
    %\Large #1
    #1
  \end{center}
}
%    \end{macrocode}
% \end{macro}

% \begin{macro}{\mpbibliography}
% \changes{1.6}{03 May 2005}
%   {Added intelligent bibliography making page}
% \changes{1.8}{09 May 2005}
%   {Changed to produce warning when no references specified.}
%  A call to the |\mpbibliograph| causes the bibliography to be produced
%  This is not for authors to call but rather for the end document and
%  |\appendix| macro to call.
%    \begin{macrocode}
\newcommand{\mpbibliography}{
  \ifmadebib\relax\else
  \ifx\undefined\@thesisbibstyle
  \typeout{WARNING: YOU NEED A BIBLIOGRAPHY!! See the references macro}
  \else
  \bibliographystyle{\@thesisbibstyle}
  \bibliography{\@thesisbibfile}
  \addcontentsline{toc}{chapter}{\bibname}
  \madebibtrue\fi
  \fi
}
%    \end{macrocode}
% \end{macro}

% \begin{macro}{\appendix}
% \changes{1.6}{03 May 2005}
%   {modified appendix to also produce bibliography}
%    Most publications have the bibliography at the very end so that it
%    is easy to find. The Graduate Evaluator's guidelines dictate that
%    the bibliography for a thesis appears prior to the appendix.
%    So a modification is made so that when an author switches chapters
%    to appendix mode the bibliography is inserted first.
%    \begin{macrocode}
\renewcommand\appendix{\mpbibliography\par
  \setcounter{chapter}{0}%
  \setcounter{section}{0}%
  \gdef\@chapapp{\appendixname}%
  \gdef\thechapter{\@Alph\c@chapter}}
%    \begin{macrocode}
% \end{macro}

% \begin{macro}{\mpproposal}
% \changes{1.6}{14 Apr 2005}
%   {Modified proposal title page typeface size to be identical with document}
%   This command creates a title page. It relies on other commands to
%   set content details prior to its invokation.
%    \begin{macrocode}
\newcommand{\mpproposal}{
  \ifx\undefined\@author
  \ClassError{CSUNthesis}{no \protect\author{<author>} given.}{}
  \fi
  \ifx\undefined\@title
  \ClassError{CSUNthesis}{no \protect\title{<title>} given.}{}
  \fi
  {
    \thispagestyle{empty}
    \begin{center}
      \vspace*{\frontmattertopmargin}
      %University heading
%      \fontsize{12}{14.4}\selectfont
      CALIFORNIA STATE UNIVERSITY, NORTHRIDGE

      %space before title
      \vspace{0.625in}

      %title
      %minipage required because the title should wrap rather narrow
      \begin{minipage}{5.5in}
	\centering
	\begin{spacing}{0.0}
%	  \fontsize{12}{14.4}\selectfont

	  \textbf{\MakeUppercase{\@title}}

	\end{spacing}

	%space before declaration
	\vspace{0.375in}

	\begin{spacing}{3.0}
%	  \fontsize{12}{14.4}\selectfont

	  \ifthesis A thesis
          \else
             \ifproject A graduate project
             \else An abstract
             \fi
          \fi
	  proposal for the degree of
	  \ifx\undefined\@degree
	    Master~of~Science
	  \else
	    \@degree
	  \fi
	  \ in
	  \ifx\undefined\@department
	    Computer~Science
	  \else
	    \@department
	  \fi

	  \vspace{0.375in}

	  By

	  \vspace{0.375in}

	  \@author
	  \ifx\undefined\@contact
	  \relax
	  \else
	  \\\@contact
	  \fi

	  \today

	  \ifx\undefined\@collaborator
	  \relax
	  \else
	  \vspace*{0.625in}
	  in collaboration with\\\@collaborator
	  \fi
	\end{spacing}
      \end{minipage}

      % the date needs to be outside of the minipage so that \vfill
      % works to flush the date to the bottom of the page.

      \vfill
      \hfill
      \begin{minipage}{4.00in}

	\makebox[2.75in]{\hrulefill}\makebox[0.5in][r]{Date:}
	\hrulefill\\
	\makebox[4.125in][l]{\@memberChair, Committee Chair}

	\vspace{.25in}

	\makebox[2.75in]{\hrulefill}\makebox[0.5in][r]{Date:}
	\hrulefill\\
	\makebox[4.125in][l]{\@coordinator, Graduate Coordinator}

	\vspace{.25in}

	\makebox[2.75in]{\hrulefill}\makebox[0.5in][r]{Date:}
	\hrulefill\\
	\makebox[4.125in][l]{\@author, Student}

      \end{minipage}
      %fill to the bottom of page
    \end{center}
  }
  \newpage
}
%    \end{macrocode}
% \end{macro}

% \begin{macro}{\mpannouncement}
%   Command to produce an announcement page. If the defense date has
%   been set then the conditional will skip creating the announcement
%   page. Announce page is not numbered and does not affect the
%   numbering of other pages.
%    \begin{macrocode}
\newcommand{\mpannouncement}
{
  \ifx\undefined\@defensedate
  \relax
  \else
  \thispagestyle{empty}
  % save current margin information
  \newlength{\oldwidth}
  \newlength{\oldoddside}
  \newlength{\oldtop}
  \newlength{\oldheight}
  \setlength{\oldtop}{\topmargin}
  \setlength{\oldheight}{\textheight}
  \setlength{\oldwidth}{\textwidth}
  \setlength{\oldoddside}{\oddsidemargin}
  % set new margins for even page layout
  \setlength{\oddsidemargin}{0.25in}
  \setlength{\textwidth}{6in}
  \setlength{\topmargin}{0.25in}
  \setlength{\textheight}{8.5in}
  % draw the simple border frame
  % picture width and height needs to be (0,0) so as not
  % to take up space. (you can draw outside the boundary
  \noindent\begin{picture}(0,0)(72,-81.5)
  \put(0,0){\line(1,0){576}}
  \put(0,0){\line(0,-1){756}}
  \put(576,-756){\line(-1,0){576}}
  \put(576,-756){\line(0,1){756}}
  \end{picture}
  \vspace{-.5in}
  \begin{center}
    \bfseries
    \begin{minipage}{5in}
      \begin{center}
	MASTERS PRESENTATION
	
	\vspace{0.5in}
	
	\MakeUppercase{\@title}
	
	By
	
	\@author
      \end{center}
      
      \noindent Committee Members:
      \begin{list}{}{\leftmargin=2in\itemsep=-6pt\topsep=-6pt}
      \item \@memberChair\ (Chair)
      \item \@memberA
      \item \@memberB
      \end{list}
      
      \vspace{14pt}
      \noindent\begin{tabular}{@{}ll}
      Date: & \@defenseday, \@defensedate\ at \@defensetime \\
      Location: & \@defenselocation
      \end{tabular}
    \end{minipage}
  \end{center}
  \begin{center}
    \bfseries ABSTRACT
  \end{center}
  \noindent \@abstract
  
  % generate the page
  \newpage
  %reset margins to original values
  \setlength{\oddsidemargin}{\oldoddside}
  \setlength{\textwidth}{\oldwidth}
  \setlength{\textheight}{\oldheight}
  \setlength{\topmargin}{\oldtop}
  \fi
}
%    \end{macrocode}
% \end{macro}

% \begin{macro}{\mptitle}
% \changes{1.6}{14 Apr 2005}
%   {Modified thesis title page typeface size to be identical with document}
%   Creates the title page for thesis, projects and abstracts.
%    \begin{macrocode}
\newcommand{\mptitle}{
  \ifx\undefined\@submitmonth
  \ClassError{CSUNthesis}{no
    \protect\submitted{<month>}{<year>}given.}{}
  \fi
  \ifx\undefined\@submityear
  \ClassError{CSUNthesis}{no
    \protect\submitted{<month>}{<year>}given.}{}
  \fi
  \ifx\undefined\@author
  \ClassError{CSUNthesis}{no \protect\author{<author>} given.}{}
  \fi
  \ifx\undefined\@title
  \ClassError{CSUNthesis}{no \protect\title{<title>} given.}{}
  \fi
  {
    \thispagestyle{empty}
    \setcounter{page}{1}
    \begin{center}
%      \vspace*{\frontmattertopmargin}
      %University heading
%      \fontsize{12}{14.4}\selectfont
      CALIFORNIA STATE UNIVERSITY, NORTHRIDGE

      %space before title
      \vspace{2.625in}

      %title
      %minipage required because the title should wrap rather narrow
      \begin{minipage}{\titlewidth}
	\centering
	\begin{spacing}{2.0}
%	  \fontsize{12}{14.4}\selectfont

	  \MakeUppercase{\@title}

	\end{spacing}

	%space before declaration
	\vspace{0.375in}

	\begin{spacing}{1.0}
%	  \fontsize{12}{14.4}\selectfont

	  \ifthesis A thesis
          \else
             \ifproject A graduate project
             \else An abstract
             \fi
          \fi
	  submitted in partial fulfillment of the
	  requirements for the degree of
	  \ifx\undefined\@degree
	    Master~of~Science
	  \else
	    \@degree\ 
	  \fi
	  in
	  \ifx\undefined\@department
	    Computer~Science
	  \else
	    \@department
	  \fi

	  \vspace{0.375in}

	  By

	  \vspace{0.375in}

	  \@author
	  \ifdraft
	  \ifx\undefined\@contact
	  \relax
	  \else
	  \\\@contact
	  \fi
	  \fi

	  \ifx\undefined\@collaborator
	  \relax
	  \else
	  \vspace*{0.625in}
	  in collaboration with\\\@collaborator
	  \fi
	\end{spacing}
      \end{minipage}

      % the date needs to be outside of the minipage so that \vfill
      % works to flush the date to the bottom of the page.

      %fill to the bottom of page
      \vfill

%      \fontsize{12}{14.4}\selectfont

      \@submitmonth\ \@submityear
    \end{center}
  }
  \newpage
}
%    \end{macrocode}
% \end{macro}

% \begin{macro}{\mpcopyright}
% \changes{1.1}{14 Apr 2005}
%   {Forced 12pt typeface for copyright page.}
% \changes{1.1}{14 Apr 2005}
%   {added copyright entry to title of contents page.}
% \changes{1.6}{03 May 2005}
%   {Modified copyright page typeface size to be identical with document}
% \changes{1.7}{04 May 2005}
%   {Conformance: copyright declaration flushed to bottom of page.}
%   Creates the copyright page if a copyright year has been provided.
%    \begin{macrocode}
\newcommand{\mpcopyright}{
  \ifx\undefined\@copyrightyear
    \relax
  \else
  \null\vfill
  \begin{center}
    %\fontsize{14}{16.8}\selectfont
%    \fontsize{12}{14.4}\selectfont
    \copyright\ Copyright\ by \@author\ \@copyrightyear\\
    All Rights Reserved
  \end{center}
%  \vfill
  \addcontentsline{toc}{chapter}{Copyright}
  \newpage
  \fi
}
%    \end{macrocode}
% \end{macro}

% \begin{macro}{\mpsignature}
% \changes{1.6}{14 Apr 2005}
%   {Modified signature page typeface size to be identical with document}
%   Creates the signature page, with committee names
%    \begin{macrocode}
\newcommand{\mpsignature}{
  \ifx\undefined\@memberA
  \ClassError{CSUNthesis}{No
    \protect\committee given; must be provided}{}
  \fi
  \ifx\undefined\@memberB
  \ClassError{CSUNthesis}{No
    \protect\committee given; must be provided}{}
  \fi
  \ifx\undefined\@memberChair
  \ClassError{CSUNthesis}{No
    \protect\committee given; must be provided}{}
  \fi
  \ifx\undefined\@author
  \ClassError{CSUNthesis}{no \protect\author{<author>} given.}{}
  \fi
  \vspace*{2.5in}
  \begin{center}
%    \fontsize{12}{14.4}\selectfont
    \begin{minipage}{5.05in}
      The
      \ifthesis
      thesis
      \else
      \ifproject
      graduate project
      \else
      abstract
      \fi
      \fi
      of \@author\ is approved:

      \vspace*{1in}

      \makebox[3in]{\hrulefill}\makebox[0.5in]{}
      \makebox[1.5in]{\hrulefill}\\
      \makebox[3in][l]{\@memberA}\makebox[0.5in]{}
      \makebox[1.5in][l]{Date}

      \vspace{.375in}

      \makebox[3in]{\hrulefill}\makebox[0.5in]{}
      \makebox[1.5in]{\hrulefill}\\
      \makebox[3in][l]{\@memberB}\makebox[0.5in]{}
      \makebox[1.5in][l]{Date}

      \vspace{.375in}

      \makebox[3in]{\hrulefill}\makebox[0.5in]{}
      \makebox[1.5in]{\hrulefill}\\
      \makebox[3in][l]{\@memberChair, Chair}\makebox[0.5in]{}
      \makebox[1.5in][l]{Date}

    \end{minipage}
    \vfill
    California State University, Northridge
  \end{center}
  \addcontentsline{toc}{chapter}{Signature page}
  \newpage
}
%    \end{macrocode}
% \end{macro}

% \begin{macro}{\mppreface}
%   Creates preface pages based on the preface content specified by
%   |\preface|.
%    \begin{macrocode}
\newcommand{\mppreface}{
  \ifx\undefined\@preface
    \relax
  \else
    \frontpagesetup{Preface}
    \@preface
    \addcontentsline{toc}{chapter}{Preface}
    \newpage
  \fi
}
%    \end{macrocode}
% \end{macro}

% \begin{macro}{\mpdedication}
%   Creates a dedication page if dedication material has been provided.
%    \begin{macrocode}
\newcommand{\mpdedication}{
  \ifx\undefined\@dedication
    \relax
  \else
    \frontpagesetup{Dedication}
    \@dedication
    \addcontentsline{toc}{chapter}{Dedication}
    \newpage
  \fi
}
%    \end{macrocode}
% \end{macro}

% \begin{macro}{\mpackknowledgement}
%   Creates an acknowledgement page if acknowledgement material has
%   been provided.
%    \begin{macrocode}
\newcommand{\mpacknowledgement}{
  \ifx\undefined\@acknowledgement
    \relax
  \else
    \frontpagesetup{Acknowledgements}
    \@acknowledgement
    \addcontentsline{toc}{chapter}{Acknowledgements}
    \newpage
  \fi
}
%    \end{macrocode}
% \end{macro}

% \begin{macro}{\mptableofcontents}
%   Creates the table of contents. A simple wrapper around the
%   \LaTeX\ command.
%    \begin{macrocode}
\newcommand{\mptableofcontents}{
%%  \newlength{\oldparskip}
%%  \setlength{\oldparskip}{\parskip}
%%  \setlength{\parskip}{0pt}
%%Graduate Evaluator Shrew, now changes her mind, again, and this
%%is no longer allowed to be listed in the Table of Contents;
%%Contrary to her original whim which made me put this in here in
%%the first place.
%%  \addcontentsline{toc}{chapter}{Table of Contents}
  \tableofcontents
%%  \setlength{\parskip}{\oldparskip}
  \newpage
}
%    \end{macrocode}
% \end{macro}

% \begin{macro}{\mplistoftables}
%   Conditionally creates the list of tables.
%    \begin{macrocode}
\newcommand{\mplistoftables}{
  \iflot
    \addcontentsline{toc}{chapter}{List of Tables}
    \listoftables
    \newpage
  \fi
}
%    \end{macrocode}
% \end{macro}

% \begin{macro}{\mplistoffigures}
%   Conditionally creates the list of figures.
%    \begin{macrocode}
\newcommand{\mplistoffigures}{
  \iflof
    \addcontentsline{toc}{chapter}{List of Figures}
    \listoffigures
    \newpage
  \fi
}
%    \end{macrocode}
% \end{macro}

% \begin{macro}{\mplofillustrations}
%   Conditionally creates the list of illustrations. (Currently a
%   command that does nothing.)
%    \begin{macrocode}
\newcommand{\mplofillustrations}{}
%    \end{macrocode}
% \end{macro}

% \begin{macro}{\mplistoflistings}
%   Conditionally creates the list of source code listings.
%    \begin{macrocode}
\newcommand{\mplistoflistings}{
  \iflol
    \renewcommand{\lstlistlistingname}{\normalfont List of Listings}
    \addcontentsline{toc}{chapter}{List of Listings}
    \lstlistoflistings
    \newpage
  \fi
}
%    \end{macrocode}
% \end{macro}

% \begin{macro}{\mpabstract}
% \changes{1.1}{14 Apr 2005}
%   {Corrected typeface size for abstract body to be equal to document.}
% \changes{1.6}{14 Apr 2005}
%   {Modified abstract page typeface size to be identical with document}
%   Formats and creates the abstract page.
%    \begin{macrocode}
\newcommand{\mpabstract}
{
  \ifx\undefined\@abstract
    \ifproposal\relax\else
      \ClassError{CSUNthesis}
		 {No \protect\abstract given; must be provided}{}
    \fi
  \fi
  {
%    \vspace*{40pt}
    {
%    \fontsize{12}{14.4}\selectfont
    \newlength{\oldbaselineskip}
    \setlength{\oldbaselineskip}{\baselineskip}
    \setlength{\baselineskip}{34pt}
    \begin{center}
      ABSTRACT
      
      \MakeUppercase{\@title}
      
      By
      
      \@author
      
      \@degree\ in \@department
    \end{center}
    }
    %\setlength{\baselineskip}{\oldbaselineskip}
		\begin{doublespace}
    \@abstract
		\end{doublespace}
  }
  \addcontentsline{toc}{chapter}{Abstract}
  \newpage
}
%    \end{macrocode}
% \end{macro}

% \begin{macro}{\AtBeginDocument}
%   When the author invokes the |\begin{document}| environment all of
%   the commands that \underline{m}ake front matter \underline{p}ages
%   (|\mp*|) are called in order to produce the properly formatted
%   front matter.  All pages creation commands are invoked and control
%   of whether or not individual pages are actually created are left
%   up to the individual page making commands.
%    \begin{macrocode}
\AtBeginDocument{
  \pagenumbering{roman}
  \ifproposal
  \mpproposal
  \else
  \parskip=10pt
  \mpannouncement
  \mptitle
  \mpcopyright
  \mpsignature
  \mpdedication
  \mpacknowledgement
  \mppreface
  \parskip=0pt
  \mptableofcontents
  \mplistoftables
  \mplistoffigures
  \mplofillustrations
  \mplistoflistings
  \parskip=10pt
  \mpabstract
  \fi
  \pagenumbering{arabic}
}
%    \end{macrocode}
% \end{macro}

% \begin{macro}{\AtEndDocument}
%   At the end of the document the bibliography is automatically
%   produced if a bibliographic style has been provided.
%    \begin{macrocode}
\AtEndDocument{%
\mpbibliography%
}
%    \end{macrocode}
% \end{macro}
%
% \label{sec:implementation_end}
% \Finale
% \obeyspaces
% \typeout{****************************************************}
% \typeout{*                                                  *}
% \typeout{* To finish the installation you have to move the  *}
% \typeout{* following file into a directory searched by Tex: *}
% \typeout{*                                                  *}
% \typeout{*       CSUNthesis.cls                             *}
% \typeout{*                                                  *}
% \typeout{* Most students will find it more convenient to    *}
% \typeout{* put the file in the same location as their       *}
% \typeout{* thesis source. (Since the current directory is   *}
% \typeout{* searched by default.) It would also be           *}
% \typeout{* acceptable to modify the TEXINPUTS environment   *}
% \typeout{* variable to include a path of the users          *}
% \typeout{* choosing.                                        *}
% \typeout{*                                                  *}
% \typeout{* To produce the documentation run the file        *}
% \typeout{* CSUNthesis.dtx through LaTeX.                    *}
% \typeout{*                                                  *}
% \typeout{****************************************************}

\endinput
